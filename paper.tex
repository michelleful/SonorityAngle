\documentclass[12pt]{article}
\usepackage[margin=1in]{geometry}
\usepackage[table]{xcolor}    %for shading*
\usepackage{fullpage}
\usepackage{natbib}
\usepackage{mathpazo}
\usepackage{tipa}
\usepackage{setspace}
\usepackage{amsmath}
\usepackage{linguex}
\usepackage{tikz}
\usetikzlibrary{matrix,arrows}
\usepackage{slashbox}	
\usepackage{pifont}    %for pointing hand
\usepackage{arydshln}    %for dashed lines
\usepackage{rotating}    %for angled text

\usepackage{amsfonts}
\newcommand{\tickYes}{\checkmark}
\usepackage{pifont}
\newcommand{\tickNo}{\hspace{1pt}\ding{55}}

\newcommand*\circled[1]{%
  \tikz[baseline=(C.base)]\node[draw,circle,inner sep=0.5pt](C) {#1};\!
}

\makeatletter
\newcommand\textsubscript[1]{\@textsubscript{\selectfont#1}}
\def\@textsubscript#1{{\m@th\ensuremath{_{\mbox{\fontsize\sf@size\z@#1}}}}}
\newcommand{\iep}{\textipa{1}}
\newcommand{\bet}{\textipa{B}}
\newcommand{\schwa}{\textipa{@}}

\newsavebox{\sonorityanglehierarchycompressed}
\savebox{\sonorityanglehierarchycompressed}{
\begin{tikzpicture}[scale=0.8,shorten >=1pt,->]
  \tikzstyle{vertex}=[circle]
  \tikzstyle{point}=[circle,fill=black!25,minimum size=12pt,inner sep=2pt]
  \tikzstyle{line} = [draw, -latex']
  \node[vertex] (RT) at (2.4*8,1)   {RT};
  \node[vertex] (NT) at (2.4*8,0.5)  {NT};
  \node[vertex] (FT) at (2.1*8,1)  {FT};
  \node[vertex] (TT) at (1.4*8,1)     {TT};
  \node[vertex] (RF) at (2.2*8,1)     {RF};
  \node[vertex] (NF) at (2.0*8,1)  {NF};
  \node[vertex] (TF) at (0.6*8,1)   {TF};
  \node[vertex] (FF) at (1.3*8,1)   {FF};
  \node[vertex] (RN) at (1.9*8,1)   {RN};
  \node[vertex] (NN) at (1.2*8,1)   {NN};
  \node[vertex] (FN) at (0.54*8,0.5)  {FN};
  \node[vertex] (TN) at (0.3*8,1)   {TN};
  \node[vertex] (RR) at (1.1*8,1)  {RR};
  \node[vertex] (NR) at (0.46*8,1.0)  {NR};
  \node[vertex] (FR) at (0.2*8,1)   {FR};
  \node[vertex] (TR) at (0.1*8,1)   {TR};
  \node[vertex] (rising) at (0.35*8,2.5) {\underline{Rising sonority}};
  \node[vertex] (level) at (1.25*8,2.5) {\underline{Level sonority}};
  \node[vertex] (falling) at (2.1*8,2.5) {\underline{Falling sonority}};
  % axis
  \node[vertex] (axisstart) at (-0.23,-0.2) {};
  \node[vertex] (axisend)   at (2.7*8,-0.2) {};
  \draw (axisstart) -- (axisend);
  \draw (0.0*8, 0) -- (0.0*8, -0.4) -- cycle;
  \draw (1.0*8, 0) -- (1.0*8, -0.4) -- cycle;
  \draw (2.0*8, 0) -- (2.0*8, -0.4) -- cycle;
  \node[vertex] (0pointlabel) at (0.0,-0.7) {0};
  \node[vertex] (1pointlabel) at (1.0*8,-0.7) {1};
  \node[vertex] (2pointlabel) at (2.0*8,-0.7) {2};
  \node[vertex] (xaxislabel) at (2.7*8,-0.5) {\textsc{SonAngle}};
  \node (leastlikely) at (2.4*8,-1.5) {Least likely to epenthesise $\rightarrow$};
  \node (mostlikely) at (0.4*8,-1.5) {$\leftarrow$ Most likely to epenthesise};
  % numbers
  \node[font=\footnotesize,opacity=0.7] (RTno) at (2.4*8, 1.5) {2.36};
  \node[font=\footnotesize,opacity=0.7] (RFno) at (2.2*8, 1.5) {2.21};
  \node[font=\footnotesize,opacity=0.7] (FTno) at (2.1*8, 0.5) {2.11};
  \node[font=\footnotesize,opacity=0.7] (NFno) at (2.0*8, 1.5) {2.03};
  \node[font=\footnotesize,opacity=0.7] (RNno) at (1.9*8, 0.5) {1.89};

  \node[font=\footnotesize,opacity=0.7] (TTno) at (1.4*8, 1.5) {1.37};
  \node[font=\footnotesize,opacity=0.7] (FFno) at (1.3*8, 0.5) {1.33};
  \node[font=\footnotesize,opacity=0.7] (NNno) at (1.2*8, 1.5) {1.25};
  \node[font=\footnotesize,opacity=0.7] (RRno) at (1.1*8, 0.5) {1.11};

  \node[font=\footnotesize,opacity=0.7] (TFno) at (0.6*8, 1.5) {0.59};
  \node[font=\footnotesize,opacity=0.7] (FNno) at (0.54*8, 0) {0.54};
  \node[font=\footnotesize,opacity=0.7] (NRno) at (0.46*8, 1.5) {0.46};
  \node[font=\footnotesize,opacity=0.7] (TNno) at (0.3*8, 1.5) {0.27};
  \node[font=\footnotesize,opacity=0.7] (FRno) at (0.2*8, 0.5) {0.22};
  \node[font=\footnotesize,opacity=0.7] (TRno) at (0.1*8, 1.5) {0.12};

\end{tikzpicture}
}

\newsavebox{\sonorityrisehierarchycompressed}
\savebox{\sonorityrisehierarchycompressed}{
\begin{tikzpicture}[scale=0.8,shorten >=1pt,->]
  \tikzstyle{vertex}=[circle]
  \tikzstyle{point}=[circle,fill=black!25,minimum size=12pt,inner sep=2pt]
  \tikzstyle{line} = [draw, -latex']
  \node[vertex] (RT) at (2.5*8, 0)  {RT};
  \node[vertex] (NT) at (1.67*8,0)  {NT};
  \node[vertex] (FT) at (1.25*8,0)  {FT};
  \node[vertex] (TT) at (1*8,-0.5)     {TT};
  \node[vertex] (RF) at (2*8,0)     {RF};
  \node[vertex] (NF) at (1.33*8,-0.5)  {NF};
  \node[vertex] (FF) at (1*8,0)   {FF};
  \node[vertex] (TF) at (0.8*8,-0.5)   {TF};
  \node[vertex] (RN) at (1.5*8,0)   {RN};
  \node[vertex] (NN) at (1*8,0.5)     {NN};
  \node[vertex] (FN) at (0.75*8,0)  {FN};
  \node[vertex] (TN) at (0.6*8,0)   {TN};
  \node[vertex] (RR) at (1*8,1.0)  {RR};
  \node[vertex] (NR) at (0.67*8,-0.5)  {NR};
  \node[vertex] (FR) at (0.5*8,0)   {FR};
  \node[vertex] (TR) at (0.4*8,0)   {TR};
  \node[vertex] (rising) at (0.5*8,2) {\underline{Rising sonority}};
  \node[vertex] (level) at (1*8,2) {\underline{Level sonority}};
  \node[vertex] (falling) at (1.67*8,2) {\underline{Falling sonority}};
  % axis
  \node[vertex] (axisstart) at (-0.23,-1.0) {};
  \node[vertex] (axisend)   at (2.7*8,-1.0) {};
  \draw (axisstart) -- (axisend);
  \draw (0.0*8, -1.2) -- (0.0*8, -0.8) -- cycle;
  \draw (1.0*8, -1.2) -- (1.0*8, -0.8) -- cycle;
  \draw (2.0*8, -1.2) -- (2.0*8, -0.8) -- cycle;
  \node[vertex] (0pointlabel) at (0.0,-1.5) {0};
  \node[vertex] (1pointlabel) at (1.0*8,-1.5) {1};
  \node[vertex] (2pointlabel) at (2.0*8,-1.5) {2};
  \node[vertex] (xaxislabel) at (2.7*8,-1.5) {\textsc{SonRise}};
\end{tikzpicture}
}

\newsavebox{\syllablecontacthierarchy}
\savebox{\syllablecontacthierarchy}{
\begin{tikzpicture}[scale=0.8,shorten >=1pt,->]
  \tikzstyle{vertex}=[circle]
  \tikzstyle{point}=[circle,fill=black!25,minimum size=12pt,inner sep=2pt]
  \tikzstyle{line} = [draw, -latex']
  \node[vertex] (RT) at (3*3,2.0)   {RT};
  \node[vertex] (NT) at (2*3,1.5)  {NT};
  \node[vertex] (FT) at (1*3,1.0)  {FT};
  \node[vertex] (TT) at (0*3,0.5)     {TT};
  \node[vertex] (RF) at (2*3,2.0)     {RF};
  \node[vertex] (NF) at (1*3,1.5)  {NF};
  \node[vertex] (FF) at (0*3,1.0)   {FF};
  \node[vertex] (TF) at (-1*3,0.5)   {TF};
  \node[vertex] (RN) at (1*3,2.0)   {RN};
  \node[vertex] (NN) at (0*3,1.5)   {NN};
  \node[vertex] (FN) at (-1*3,1)  {FN};
  \node[vertex] (TN) at (-2*3,0.5)   {TN};
  \node[vertex] (RR) at (0*3,2.0)  {RR};
  \node[vertex] (NR) at (-1*3,1.5)  {NR};
  \node[vertex] (FR) at (-2*3,1.0)   {FR};
  \node[vertex] (TR) at (-3*3,0.5)   {TR};
  % headers
  \node[vertex] (rising) at (2*3,2.6) {\underline{Falling sonority}};
  \node[vertex] (level) at (0*3,2.6) {\underline{Level sonority}};
  \node[vertex] (falling) at (-2*3,2.6) {\underline{Rising sonority}};
  % axis
  \node[vertex] (axisstart) at (-3.5*3,-0) {};
  \node[vertex] (axisend)   at (3.5*3,-0) {};
  \draw (axisstart) -- (axisend);
  \draw (0, 0.2) -- (0, -0.2) -- cycle;
  \draw (1*3, 0.2) -- (1*3, -0.2) -- cycle;
  \draw (2*3, 0.2) -- (2*3, -0.2) -- cycle;
  \draw (3*3, 0.2) -- (3*3, -0.2) -- cycle;
  \draw (-1*3, 0.2) -- (-1*3, -0.2) -- cycle;
  \draw (-2*3, 0.2) -- (-2*3, -0.2) -- cycle;
  \draw (-3*3, 0.2) -- (-3*3, -0.2) -- cycle;
  \node[vertex] (0pointlabel) at (0.0,-0.5) {0};
  \node[vertex] (1pointlabel) at (1.0*3,-0.5) {1};
  \node[vertex] (2pointlabel) at (2.0*3,-0.5) {2};
  \node[vertex] (3pointlabel) at (3.0*3,-0.5) {3};
  \node[vertex] (-1pointlabel) at (-1.0*3,-0.5) {-1};
  \node[vertex] (-2pointlabel) at (-2.0*3,-0.5) {-2};
  \node[vertex] (-3pointlabel) at (-3.0*3,-0.5) {-3};
  \node[vertex] (xaxislabel) at (3.5*3,-0.5) {\textsc{Dis}};
\end{tikzpicture}
}





\title{The perceptual dimensions of \\ sonority-driven epenthesis}
\author{Michelle A. Fullwood}
\date{Generals Paper, MIT \\ September 2013}
\begin{document}

\maketitle

\begin{abstract}
 
Vowel epenthesis often appears to preferentially target consonant clusters with rising sonority.
One explanation for this is perceptual faithfulness \citep{fleischhacker.2002,steriade.2006}: rising sonority clusters are more susceptible to epenthesis because the perceptual distance between the underlying /C\textsubscript{1}C\textsubscript{2}/ sequence and its correspondent output sequence [C\textsubscript{1}VC\textsubscript{2}] is small, thus incurring a smaller faithfulness cost.
This raises the question of how to compute the perceptual distance between two sonority contours /C\textsubscript{1}C\textsubscript{2}/ and [C\textsubscript{1}VC\textsubscript{2}] in terms of the sonority of C\textsubscript{1}, C\textsubscript{2} and V.  
In this paper, I propose that the appropriate metric is {\sc Sonority Angle}, the angle formed by the contours C\textsubscript{1}C\textsubscript{2} and C\textsubscript{1}V, and apply it in analyzing two case studies of sonority-driven epenthesis, Chaha and Irish.  A comparison is made to another possible metric,
{\sc Sonority Rise} \citep{flemming.2008}, the ratio of the gradients of the two contours, as well as to Syllable Contact, which represents an alternative, markedness-based approach to the problem of sonority-driven epenthesis. 
\end{abstract}

\newpage
\tableofcontents
\newpage

\section{Introduction}

Vowel epenthesis often appears to preferentially target consonant clusters with rising sonority.
There are two broad classes of explanation within Optimality Theory for such sonority-driven epenthesis.

One is faithfulness-based: the perceptual distance between the underlying /C\textsubscript{1}C\textsubscript{2}/ sequence and its correspondent
output sequence [C\textsubscript{1}VC\textsubscript{2}] is small when the cluster is of rising sonority.
Thus, epenthesis into such a sequence incurs a smaller faithfulness cost than epenthesis into a cluster of falling sonority. This is the basis of the analysis proposed by Fleischhacker (2002, 2005) to explain why rising sonority obstruent-sonorant clusters are more easily epenthesised in
to than falling sonority sibilant-stop clusters.

This faithfulness-based approach raises the question of how the perceptual distance between two sonority contours /C\textsubscript{1}C\textsubscript{2}/ and [C\textsubscript{1}VC\textsubscript{2}] should be computed in terms of the sonority of C\textsubscript{1}, C\textsubscript{2} and V.  
Fleischhacker's analysis rested on empirical determinations of sonority contour faithfulness, and did not attempt to determine such a relation.  

\citet{steriade.2006} proposed that input and output sonority contours should match in terms of whether they are rising or falling, and to what degree, but did not suggest a concrete mathematical relation.  \citet{flemming.2008} formalises Steriade's approach with the metric {\sc Sonority Rise}, the ratio of the gradients of the two contours.

\bigskip

In this paper, I suggest an alternative metric, {\sc Sonority Angle}, namely the magnitude of the angle made by the vectors C\textsubscript{1}-C\textsubscript{2} and C\textsubscript{1}-V, and explore the ramifications of this choice. 

{\sc Sonority Angle} makes the same broad predictions as {\sc Sonority Rise} -- that clusters of rising sonority, having a relatively small angle between the underlying sonority contour /C\textsubscript{1}-C\textsubscript{2}/ and the overt sonority contour [C\textsubscript{1}-V], are perceptually more similar to their epenthetic output, and therefore more likely to undergo epenthesis, than clusters of falling sonority. Crucially, however, the exact hierarchy of susceptibility of individual clusters to epenthesis is predicted to be different.

I take two instances where the predictions of {\sc Sonority Angle} and {\sc Sonority Rise} differ and illustrate with case studies of sonority-driven epenthesis in two different languages, namely Chaha and Irish, that the predictions of {\sc Sonority Angle} are more in line with the data than those of {\sc Sonority Rise}.

\bigskip

The other broad class of explanation for sonority-driven epenthesis is markedness-based.  Syllable Contact \citep{murray.vennemann.1983} holds that across a syllable boundary, falling sonority clusters are more harmonic than rising sonority ones.  Hence, rising sonority clusters are preferentially broken up by epenthesis.

Syllable Contact forms the basis for the main existing analysis of Chaha epenthesis by \citet{rose.2000}.  I show that the faithfulness-based analysis, powered by the metric of {\sc Sonority Angle}, is more economical.  In the case of Irish, Syllable Contact makes incorrect predictions regarding the data.

\bigskip

The layout of this paper is as follows.  Section \ref{theoreticalmachinery} lays out the theoretical background for the sonority contour faithfulness approach to 
sonority-driven epenthesis. I introduce the proposed {\sc Sonority Angle} metric as well as the competing {\sc Sonority Rise} metric \citep{flemming.2008}, 
then lay out the alternative markedness-based approach to sonority-driven epenthesis, namely {\sc Syllable Contact}.

Section \ref{irish} consists of a case study of sonority-driven epenthesis in Irish.  I show that the data are in line with the predictions of {\sc Sonority Angle} and not {\sc Sonority Rise}, while a Syllable Contact-based analysis would have to be very complicated to explain the same facts.

Section \ref{chaha} is a case study of epenthesis positioning in Chaha.  I detail the facts of epenthesis positioning in Chaha, based on the data given in \citet{rose.2000},
and show that the sonority contour faithfulness approach explains these facts, with {\sc Sonority Angle} as the metric for comparing sonority contours.  I compare it to {\sc Sonority Rise} and show that the former is the more successful analysis, and that overall, the approach just outlined is more economical than the 
Syllable Contact-based approach of \citet{rose.2000}.

Section \ref{issues} looks at various issues regarding {\sc Sonority Angle}, such as its robustness.
Section \ref{conclusion} concludes.

\section{Theoretical background} \label{theoreticalmachinery}

This paper assumes as its basis the P-map hypothesis \citep{steriade.2001}, which states that the perceptual distance between underlying representations and potential surface forms projects a fixed ranking of faithfulness constraints.

In order to determine what faithfulness constraints exist in {\sc Con} and what their rankings should be, therefore, we need to know the metrics of perceptual distance that are relevant to each change. In the case of vowel epenthesis, the perceptual distance to be measured is between two sonority contours, /C1-C2/ and [C\textsubscript{1}-V-C\textsubscript{2}].

\subsection{Sonority Angle}

The observation with which we started was that the more steeply rising the sonority profile of a consonant cluster, the more likely the cluster to undergo epenthesis. Thus the absolute difference in sonority between C\textsubscript{1} and C\textsubscript{2} must be factored into the metric. To this, I add the claim that the more sonorous C\textsubscript{1}, the more likely the cluster is to undergo epenthesis.

These two factors are neatly captured by the metric {\sc Sonority Angle}, which is defined as the angle between the underlying C\textsubscript{1}C\textsubscript{2} sonority contour and the surface C\textsubscript{1}V contour:

\ex. \label{sonangle_picture} \begin{tikzpicture} [shorten >=1pt,scale=0.33]
                      \draw [<-] (-3,6.5) -- (-3, 0.5) ; % axis
                      \node at (-3, 7.0) {Sonority}; % axis label
                      \draw (5,1) -- (0,3) -- (5,6) ; % rising sonority
                      \node at (1.8, 3.2) {$\theta$}; % theta label for rising sonority
                      \node[left] at (0,3) {C$_1$}; 
                      \node[right] at (5,1) {C$_2$};
                      \node[right] at (5,6) {V};
    \coordinate (A) at (0,3);
    \coordinate (B) at (5,1);
    \coordinate (C) at (5,6);
\end{tikzpicture} 

Assuming that the horizontal distance is 1 unit, we can compute the magnitude of this angle analytically with the following formula:

\ex. \label{sonangle_formula} Formula: \textsc{SonAngle} = $arctan(V-C_1) - arctan(C_2-C_1)$

Let us verify that {\sc Sonority Angle} does indeed reflect the two generalisations we wish to make:
first, that the smaller the sonority distance between C\textsubscript{1} and C\textsubscript{2},
the smaller the sonority angle.
Imagine fixing C\textsubscript{1} as in \ref{sonangle_picture} and raising the sonority of C\textsubscript{2}. Intuitively, this decreases the {\sc Sonority Angle}, comparing the two below.

\ex. \begin{tikzpicture} [shorten >=1pt,scale=0.33]
                      \draw [<-] (-3,6.5) -- (-3, 0.5) ; % axis
                      \node at (-3, 7.0) {Sonority}; % axis label
					 % left diagram
                      \draw (5,1) -- (0,3) -- (5,6) ; % rising sonority
                      \node at (1.8, 3.2) {$\theta$}; % theta label for rising sonority
                      \node[left] at  (0,3) {C$_1$}; 
                      \node[right] at (5,1) {C$_2$};
                      \node[right] at (5,6) {V};
					  % right diagram
                      \draw (20,4) -- (15,3) -- (20,6) ; % falling sonority
                      \node at (17.5, 4.0) {$\theta$}; % theta label for falling sonority
                      \node[left] at (15,3) {C$_1$}; 
                      \node[right] at (20,4) {C$_2$};
                      \node[right] at (20,6) {V};
\end{tikzpicture} 

The dependence of {\sc Sonority Angle} on this distance can also be seen in the second term in \ref{sonangle_formula}.

The second generalisation is that the more sonorous the C\textsubscript{1}, the more likely
epenthesis is to occur. This time, fix C\textsubscript{2} and lower the sonority of C\textsubscript{1}.

\ex. \begin{tikzpicture} [shorten >=1pt,scale=0.33]
                      \draw [<-] (-3,6.5) -- (-3, 0.5) ; % axis
                      \node at (-3, 7.0) {Sonority}; % axis label
					  % left diagram
                      \draw (5,1) -- (0,3) -- (5,6) ; % rising sonority
                      \node at (1.8, 3.2) {$\theta$}; % theta label for rising sonority
                      \node[left] at  (0,3) {C$_1$}; 
                      \node[right] at (5,1) {C$_2$};
                      \node[right] at (5,6) {V};
					  % right diagram
                      \draw (20,1) -- (15,1) -- (20,6) ; % falling sonority
                      \node at (17, 1.9) {$\theta$}; % theta label for falling sonority
                      \node[left] at (15,1) {C$_1$}; 
                      \node[right] at (20,1) {C$_2$};
                      \node[right] at (20,6) {V};
\end{tikzpicture} 

While the difference is less clear visually, the second angle is larger. The first term in the formula confirms the relation between the sonority of C\textsubscript{1} in terms of its closeness to V, and {\sc Sonority Angle} as a whole.

Given a sonority scale where classes of consonants are mapped to a numerical sonority, we can now calculate the {\sc Sonority Angle} for any cluster, which can be thought of as the faithfulness cost of epenthesising between the two consonants.  Examples of the calculation are given below.

In this paper, I adopt (with, later, minor modifications) the following standard scale:

\ex. \label{standardsonorityscale} 
      \begin{tabular}{cccccc}
         T & F & N & R & G & V \\
         stop & fricative & nasal & liquid & glide & vowel \\
         1 & 2 & 3 & 4 & 5 & 6 \\
      \end{tabular}

The {\sc Sonority Angles} for NT, TT and TN are calculated as in the following examples.

\ex. \a. \textsc{SonAngle}(NT) =  $arctan(6-3) - arctan(1-3)$ = 2.35
     \b. \textsc{SonAngle}(TT) =  $arctan(6-1) - arctan(1-1)$ = 1.37
     \c. \textsc{SonAngle}(TN) =  $arctan(6-1) - arctan(3-1)$ = 0.27

The larger the {\sc Sonority Angle}, the larger the faithfulness cost. We therefore expect it to be hardest to epenthesise into NT out of these three clusters, and easiest to epenthesise into TN.

We formalise the idea of the faithfulness cost by defining a family of \textsc{Ident} constraints that penalise outputs that incur faithfulness costs of greater than a certain $n$.

\ex. \textsc{Ident(SonAngle)}$<n$: Assign a violation mark if the consonants in two strings C\textsubscript{1}C\textsubscript{2} and C\textsubscript{1}VC\textsubscript{2} 
stand in correspondence, and the sonority angle between C\textsubscript{1}C\textsubscript{2} and C\textsubscript{1}V is greater than $n$. % TODO: change to it being a property of C1?

These faithfulness constraints have a universal ranking, with the least stringent the highest-ranked.

   \ex. ... \newline
            $\gg$ \textsc{Ident(SonAngle)}$<$1.5 \newline
            $\gg$ \textsc{Ident(SonAngle)}$<$1.0 \newline
            $\gg$ \textsc{Ident(SonAngle)}$<$0.5 \newline
            $\gg$ ...

The resulting hierarchy of clusters, ranked according to their resistance to epenthesis as defined
by their {\sc Sonority Angle}, is as follows.

\ex. {\sc Sonority Angle} hierarchy

\vspace{-3em}
\noindent \resizebox{\linewidth}{!}{\usebox{\sonorityanglehierarchycompressed}}

Notice that out of the falling sonority clusters, those that decrease in sonority by a single step
-- namely RN, NF and FT -- have smaller {\sc Sonority Angles} than the ones that have a greater fall
in sonority. Furthermore, out of these three clusters, theone with the most sonorous C\textsubscript{1}, RN, has the smallest {\sc Sonority Angle}. We thus predict that out of the falling sonority clusters,
RN is the most likely to be broken up by epenthesis. The case study on Chaha will demonstrate that this
is the case.

Similarly, between the clusters that fall in sonority by two steps -- RF and NT -- we expect
NT to be less likely to undergo epenthesis, since N is less sonorous than R. We expect NT and RT to be the clusters most resistant to epenthesis out of all the clusters. This will be crucial 
to our analysis of Irish sonority-driven epenthesis.

\subsection{Sonority Rise}

I will contrast {\sc Sonority Angle} with an alternative metric of sonority contour faithfulness proposed by \citet{flemming.2008}, {\sc Sonority Rise}, which takes the ratio of the underlying sonority contour /C\textsubscript{1}C\textsubscript{2}/ and the surface contour [C\textsubscript{1}V]. 

   \ex. \textsc{Sonority Rise} = $1-\frac{C_2-C_1}{V-C_1}$
   
The following sample calculations illustrate how \textsc{SonRise} distance is computed:

\ex.    \begin{center}
    \begin{tabular}{c c | c c | c}
    C\textsubscript{1}C\textsubscript{2}   & Rise & C\textsubscript{1}V & Rise & \textsc{SonRise} Distance\\ \hline
    NT &  -2  &  N\textipa{1}T & 3 & $1-\frac{1-3}{6-3}=1.7$ \\
    TT &  0   &  T\textipa{1}T & 5 & $1-\frac{1-1}{6-1}=1.0$ \\
    TN &  2   &  T{\textipa{1}}N & 5 & $1-\frac{3-1}{6-1}=0.6$ \\ 
    \end{tabular}
    \end{center}

As with {\sc Sonority Angle}, this gives rise to a family of constraints {\sc Ident(SonRise)}$<n$, defined similarly to {\sc Ident(SonAngle)}$<n$.

{\sc Sonority Rise} also gives rise to a hierarchy of susceptbility of clusters.

\ex. {\sc Sonority Rise} hierarchy

\vspace{-3em}
\noindent \resizebox{\linewidth}{!}{\usebox{\sonorityrisehierarchycompressed}}

Though in many ways similar to the {\sc Sonority Angle} hierarchy, it makes several crucially different predictions. For example, it does not share the prediction of {\sc Sonority Angle} that RN is the most likely of the falling sonority clusters to epenthesise -- rather, if RN undergoes epenthesis then we expect FT and NF to do the same. Furthermore, it predicts that if NT and RT fail to undergo epenthesis due to the high faithfulness cost of interrupting these clusters, then RF should also fail to undergo epenthesis. Both of these predictions are contrary to the evidence of Irish and Chaha.

\subsection{Another dimension of sonority contour faithfulness} \label{depson}

Note that I do not claim that sonority contour faithfulness is the only dimension of perceptual similarity relevant to vowel epenthesis. For instance, I will adopt \citet{flemming.2008}'s suggestion that a further dimension of perceptual similarity restricts epenthesis to sites adjacent to a sonorant, based on patterns of epenthesis in languages such as Montana Salish, where epenthesis is disallowed between two obstruents. 

I will formalise this restriction with the constraint {\sc Dep(+sonorant)}. The epenthesis of a vowel between two obstruents, which bear the feature [$-$sonorant], necessitates the insertion of a new [+sonorant] feature, whereas when a vowel is epenthesised adjacent to a sonorant, the sonorant's [+sonorant] feature spreads and is shared between the vowel and the sonorant.

This constraint will be used in the analysis of Chaha complex coda epenthesis, which in some idiolects
cannot occur between obstruents, while in others, it can. My proposal will be that in the former, {\sc Dep(+sonorant)} is higher-ranked, thus blocking epenthesis.

\subsection{Syllable Contact} 

The alternative markedness-based approach to sonority-driven epenthesis that I will explore in this paper is Syllable Contact, which was first stated as the Syllable Contact Law by \citet{murray.vennemann.1983}:

\ex. ``The preference for a syllabic structure $A$\$$B$, where $A$ and $B$ are marginal segments and $a$ and $b$ are the Consonantal Strength
values of $A$ and $B$ respectively, increases with the value of $b$ minus $a$'' \citep{murray.vennemann.1983}

\citet{rose.2000} reformulates this as a violable, but categorical, constraint within the context of Optimality Theory and uses it in an analysis of Chaha sonority-driven epenthesis:

\ex.  \textsc{SyllCon}: The first segment of the onset of a syllable must be lower in sonority than the last segment in the immediately preceding syllable.

More recently, Syllable Contact has been recast as a gradient family of constraints, for example by \citet{gouskova.2002, gouskova.2004}.  She defines the distance {\sc Dis} between two consonants in a syllable contact situation as the sonority of the second minus the sonority of the first.  For example, [t.s] = +1 if [s] and [t] are only one step apart on a sonority scale.  She then defines the following family of constraints:

\ex. {\sc *Dis-}$n$: Assign a violation mark if the {\sc Dis} between two adjacent heterosyllabic consonants is $n$ (adapted from \citep{gouskova.2002})

This gives rise to a universal hierarchy:

\ex. {\sc *Dis+7} $\gg$ {\sc *Dis+6} $\gg$ ... {\sc *Dis+0} $\gg$ {\sc *Dis-1} $\gg$ ... {\sc *Dis-7}.

Thus heterosyllabic clusters that rise sharply in sonority -- that have a high {\sc Dis} -- are more marked than more falling clusters.

The predicted hierarchy of susceptibility to epenthesis of the clusters is as follows:

\ex. Syllable Contact hierarchy (based on {\sc *Dis}):

\vspace{-3em}
\noindent \resizebox{\linewidth}{!}{\usebox{\syllablecontacthierarchy}}

\bigskip

Syllable Contact applies only across syllable boundaries.  Therefore, when sonority-driven epenthesis occurs in onsets or codas, other sonority-based markedness constraints must be employed.  One such is {\sc Sonority Sequencing} \citep{selkirk.1984}, which states that sonority must be strictly increasing in the onset and strictly decreasing in the coda. \citep{rose.2000} uses two variants of this constraint, one strict and one non-strict, in her analysis of Chaha.

\section{Case study: Irish} \label{irish}

Irish displays an unusual process of epenthesis that targets consonant clusters 
whose first member is a sonorant \citep{carnie.1994, ni.chiosain.1999}, with the 
exception of sonorant-voiceless stop clusters, which never undergo epenthesis. 
I will show in this section that a {\sc Sonority Angle}-based faithfulness
neatly captures these facts.

\subsection{Data}

Irish epenthesis targets sonorant-initial clusters and occurs both across syllable boundaries and in codas.  The epenthetic vowel is [\textipa{@}] in nonpalatalised environments and [i] otherwise \citep{ni.chiosain.1999}.

\ex. \a. /alb\textipa{@}/ $\rightarrow$ [al\textipa{@b@}] \\
         {\it Alba} `Scotland'
     \b. /dorx\textipa{@}/ $\rightarrow$ [dor\textipa{@x@}] \\
         {\it dorcha} `dark'
     \b. /gorm/  $\rightarrow$  [gor\textipa{@}m] \\
         {\it gorm} `blue'
     \b. /banb\textipa{@}/ $\rightarrow$ [ban\textipa{@b@}] \\
         {\it Banba} `a name for Ireland'
     \b. /kon\textipa{f@}/ $\rightarrow$ [kon\textipa{@f@}] \\
         {\it confadh} `anger'
     \b. /an\textsuperscript{j}m\textsuperscript{j}/ $\rightarrow$ [an\textsuperscript{j}im\textsuperscript{j}] \\
         {\it ainm} `name'
     \z.
     \citep[(37)]{carnie.1994} and \citep[(2)]{ni.chiosain.1999}

The pattern of epenthesis outlined above does not apply to onset clusters.  Sonorant-initial clusters
are attested in onsets.

\ex. \a. [\textipa{mna:}] or [\textipa{mra:}] (depending on dialect) \\
         {\it mn\'a} `women'
     \b. [\textipa{m\textsuperscript{j}r\textsuperscript{j}i:ra:n}] \\
         {\it mriath\'an} `sea-rods'
     \z.
     \citep[456]{swingle.1992}

There are two classes of exceptions to this process.  First, homorganic clusters are not broken up by epenthesis.

\ex. \a. /farno\textipa{:g}/ $\rightarrow$ [farno\textipa{:g}], *[far\textipa{@}no\textipa{:g}] \\
         {\it fearnog} `alder'
     \b. /bord/ $\rightarrow$ [bord], *[bor\textipa{@}d] \\
         {\it bord} `table'
     \b. /kalr\textipa{@}/ $\rightarrow$ [kalr\textipa{@}], *[kal\textipa{@r@}] \\
         {\it calra} `calorie'  
     \z. 
     \citep[(52)]{carnie.1994}\footnote{\citet{ni.chiosain.1999} has a more complicated set of facts as to which homorganic clusters are broken up by epenthesis. As this is not the focus of this paper, I will assume the simpler dataset that Carnie presents.}

Second, epenthesis does not apply when the second consonant is a voiceless stop.

\ex. \a. /kork/ $\rightarrow$ [kork], *[kor\textipa{@}k] \\
         {\it Cork} `Cork (place name)' 
     \b. /kork\textipa{@}/ $\rightarrow$ [kork\textipa{@}], *[kor\textipa{@k@}] \\
         {\it corca} `people' 
     \b. /korp/  $\rightarrow$ [korp], *[kor\textipa{@p}] \\
         {\it corp} `body' %TODO : get some NT data in here!
     \z.
     \citep[(39a,b,c)]{carnie.1994}

\subsection{Analysis}

The fact that this process of epenthesis does not target obstruent-initial clusters is not problematic under the sonority contour faithfulness approach,
as the {\sc Ident(SonAngle)} constraints limit epenthesis, rather than trigger it.  By choosing a more limited markedness constraint such as {\sc *Son-C},
defined below, we can have epenthesis act only on sonorant-initial clusters.

\ex. {\sc *Son-C}: Assign a violation mark for every sonorant consonant preceding another consonant.

\ex. {\sc *Son-C} $\gg$ {\sc Dep}

\vspace{-2em}
\begin{center} \renewcommand*\arraystretch{1.2}
\scalebox{1}[1]{\begin{tabular}[t]{|rrl||c|c|} \hline 
\multicolumn{3}{|c||}{Input:~/\textipa{gorm}/} & {\sc *Son-C} & {\sc Dep} \\[0.5ex]
\hline \hline a. & \ding{43} & \textipa{gor@m} & & \cellcolor{lightgray}$\ast$ \\
\hline b. & & \textipa{gorm} & $\ast$! & \cellcolor{lightgray} \\
\hline \end{tabular}} \renewcommand*\arraystretch{1} \end{center}

Epenthesis does not break up onset sonorant-initial clusters.  This is likely due to the fact that
with a small number of lexical exceptions, Irish words are stress-initial \citep[26]{o.siadhail.1989}.\footnote{With the exception of Munster Irish, where stress is attracted to syllables containing
a long vowel or diphthong \citep{green.1996},
but where onset clusters still are not broken up by
epenthesis.  In this case, one would probably have to posit a positional faithfulness constraint
protecting the word-initial cluster.}  
Breaking up the onset cluster would result in an unstressable epenthetic vowel occupying the 
preferred location of stress in Irish.

I will assume initial stress is due to the following constraint, which is dominated only by 
lexicon-specific constraints.

\ex. {\sc Stress-Initial}: Assign a violation mark if the initial syllable of a word is not stressed.

\citet{alderete.2000} proposes a general constraint {\sc Head-Dependence} against making an epenthetic segment a prosodic head, which I adopt and state informally here:

\ex. {\sc Head-Dep}: Assign a violation mark for every stressed vowel not present in the input.

Together, these two constraints ranking above {\sc *Son-C} predict that onset epenthesis does not occur.

\ex. {\sc Stress-Initial}, {\sc Head-Dep} $\gg$ {\sc *Son-C}

\vspace{-2em}
\begin{center} \renewcommand*\arraystretch{1.2}
\scalebox{1}[1]{\begin{tabular}[t]{|rrl||c:c|c|} \hline 
\multicolumn{3}{|c||}{Input:~/\textipa{mna:}/} & {\sc Stress-Initial} & {\sc Head-Dep} & {\sc *Son-C} \\[0.5ex]
\hline \hline a. & \ding{43} & \textipa{"mna:} & & & \cellcolor{lightgray}$\ast$ \\
\hline b. & & \textipa{"m@.na:} & & $\ast$! & \cellcolor{lightgray} \\
\hline c. & & \textipa{m@."na:} & $\ast$! & & \cellcolor{lightgray} \\
\hline \end{tabular}} \renewcommand*\arraystretch{1} \end{center}

In codas and across syllable boundaries, epenthesis does not apply to homorganic clusters.  
I suggest that this is due to the action of the Obligatory Contour Principle (OCP).  
I adopt \citet{walters.2007}'s
formulation of the OCP, which bans successive identical oral gestures.  A cluster such as [nt] would consist of a single oral constriction gesture at the alveolar ridge, and would not constitute a violation of the OCP. Splitting it by epenthesis to form [n\textipa{@}t] would require a repeated identical constriction gesture, and therefore be a violation of the OCP.

\ex. OCP $\gg$ {\sc *Son-C}

\vspace{-2em}

\begin{center} \renewcommand*\arraystretch{1.2}
\scalebox{1}[1]{\begin{tabular}[t]{|rrl||c|c|} \hline 
\multicolumn{3}{|c||}{Input:~/\textipa{bord}/} & OCP & {\sc *Son-C} \\[0.5ex]
\hline \hline a. & \ding{43} & \textipa{bord} & & \cellcolor{lightgray}$\ast$ \\
\hline b. & & \textipa{bor@d} & $\ast$! & \cellcolor{lightgray} \\
\hline \end{tabular}} \renewcommand*\arraystretch{1} \end{center}

Lastly, we come to the sonority-driven aspect of the analysis.  
Epenthesis is allowed when C1 is ``followed by a voiced stop, fricative, nasal or liquid, but not when it is followed by a voiceless stop.'' \citep{carnie.1994}

This requires a revision to the standard sonority scale, splitting up voiceless and voiced stops: 

\ex.  \begin{center}
   \begin{tabular}{cccccccc}
    Voiceless stop & Voiced stop  & Fricative    & Nasal        & Liquid & Glide        & Vowel \\
      T            &     D       &          F    &    N &    R   &   G          & V     \\
      1            &     1.5     &          2   &    3  &    4   &   5          & 6     \\
     \end{tabular}
 \end{center}

Now let us calculate the sonority angles for the relevant set of sonorant-initial clusters in Irish.

\ex. {\sc Sonority Angles} of sonorant-initial clusters using revised sonority scale

  \begin{tikzpicture}[shorten >=1pt,->,scale=0.5]
     \tikzstyle{line} = [draw]%, -latex']

        \node (NT) at (2.48619449019234 * 15, 0) {NT}; % 2.35619449019234 - repositioning for visual clarity
        \node (ND) at (2.28183949564558 * 15, 0) {ND}; % 2.23183949564558 
        \node (NF) at (2.03444393579570 * 15, 0) {NF};
        \node (NN) at (1.24904577239825 * 15, 0) {NN};
        \node (NR) at (0.60 * 15, 0) {NR}; % 0.4636476090008

        \node (RT) at (2.48619449019234 * 15, 0.7) {RT};% 2.35619449019234 - repositioning for visual clarity
        \node (RD) at (2.37743866747662 * 15, 0) {RD}; % 2.29743866747662 
        \node (RF) at (2.19429743558818 * 15, 0) {RF}; % 2.21429743558818 
        \node (RN) at (1.89254688119154 * 15, 0) {RN};
        \node (RR) at (1.10714871779409 * 15, 0) {RR};

    \node (axisstart) at (0.5 * 15,-1) {};
    \node (axisend)   at (2.6 * 15,-1) {};
    \draw (axisstart) -- (axisend);
   \node (xaxislabel) at (2.6 * 15,-2.5) {\textsc{Sonority Angle}};

  \node (dividinglinestart) at (2.43 * 15,2) {};
  \node (dividinglineend)   at (2.43 * 15,-2) {};
  \path [line,dashed] (dividinglinestart) -- (dividinglineend) -- cycle;

  \node (2pointlabel) at (2.43 * 15,-1.5) {2.3};
        

    \end{tikzpicture} %TODO : draw arrow and mark 1.0, 1.5, 2.0, 2.5

\bigskip

The sonorant-voiceless stop clusters can be neatly separated from the rest of the clusters in terms of {\sc Sonority Angle}.
The correct constraint ranking is therefore the following:

\ex. {\sc Ident(SonAngle)$<$2.3} $\gg$ {\sc *Son-C} $\gg$ {\sc Ident(SonAngle)$<$2.4}

\begin{center} \renewcommand*\arraystretch{1.2}
\scalebox{1}[1]{\begin{tabular}[t]{|rrl||c|c|c|} \hline 
\multicolumn{3}{|c||}{Input:~/\textipa{kork}/} & {\sc Ident(SonAngle)$<$2.3} & {\sc *Son-C} & {\sc Ident(SonAngle)$<$2.4} \\[0.5ex]
\hline \hline a. & \ding{43} & \textipa{kork} & & \cellcolor{lightgray}$\ast$ & \cellcolor{lightgray} \\
\hline b. & & \textipa{kor@k} & 2.35 $\ast$! & \cellcolor{lightgray} & \cellcolor{lightgray}2.35 $\ast$ \\
\hline \end{tabular}} \renewcommand*\arraystretch{1} \end{center}

\begin{center} \renewcommand*\arraystretch{1.2}
\scalebox{1}[1]{\begin{tabular}[t]{|rrl||c|c|c|} \hline 
\multicolumn{3}{|c||}{Input:~/\textipa{alb@}/} & {\sc Ident(SonAngle)$<$2.3} & {\sc *Son-C} & {\sc Ident(SonAngle)$<$2.4} \\[0.5ex]
\hline \hline c. & \ding{43} & \textipa{al@b@} & 2.29 & & \cellcolor{lightgray} 2.29 \\
\hline d. & & \textipa{alb@} & & $\ast$! & \cellcolor{lightgray} \\
\hline \end{tabular}} \renewcommand*\arraystretch{1} \end{center}

A note on the nasal-voiceless stop clusters is in order here: such clusters tend to be homorganic on the surface.  
It is therefore unclear whether such clusters failed to be broken up by epenthesis because of OCP, 
or because of the large sonority contour faithfulness violation.  
I argue that in general, it is for the second reason.

Suppose nasal-voiceless stop clusters were banned from epenthesising because of the OCP.
Then we would expect to see surface
sequences of [NVP] where N and P have different places of articulation and V is an epenthetic vowel, 
just as we see [NVB] sequences:

\ex. \a. [\textipa{b\textsuperscript{j}in\textsuperscript{j}ib}] \\
         {\it binb} `venom'
     \b. [\textipa{ban@b@}] \\
         {\it Banba} `Banba (a name for Ireland)'
     \z.
     \citep[2c]{carnie.1994}

To my knowledge, no [NVP] sequences with the properties stated above exist.

Further, albeit marginal, evidence comes from the existence of words with heterorganic [NP] sequences,
which are not broken up by epenthesis.

\ex. \a. [\textipa{s\textsuperscript{j}e:mt}] \\
         {\it s\'eimt} `the act of playing music', dialectal variant of {\it seinm})
     \b. [\textipa{k2r\textsuperscript{j}@m\textsuperscript{j}k\textsuperscript{j}}] \\
         {\it cuirimc} `I observe, pay attention to' (Connaught dialect) \\
     (Words found in \citet{dineen.2007})

I conclude that in general, epenthesis does not occur in /NT/ sequences due to the large {\sc Sonority Angle} between NT and NVT, and not to the non-existence of underlying heterorganic /NP/ clusters in the lexicon. The epenthetic facts of Irish thus support the ranking of NT and RT clusters at the bottom of the hierarchy of susceptibility to epenthesis, which is consistent with the {\sc Sonority Angle} metric.

\subsection{Comparison to other approaches}

Using the same revised sonority scale, the {\sc Sonority Rise} calculations are as follows for the relevant clusters:

\ex. {\sc Sonority Rises} of sonorant-initial clusters using revised sonority scale

 \begin{center}
  \begin{tikzpicture}[shorten >=1pt,->,scale=0.5]
     \tikzstyle{line} = [draw]%, -latex']

        \node (NT) at (1.66666666666667 * 15, 0) {NT}; 
        \node (ND) at (1.5 * 15, 0) {ND}; 
        \node (NF) at (1.33333333333333 * 15, 0) {NF};
        \node (NN) at (1 * 15, 0) {NN};
        \node (NR) at (0.666666666666667 * 15, 0) {NR}; 

        \node (RT) at (2.5 * 15, 0) {RT};
        \node (RD) at (2.25 * 15, 0) {RD}; 
        \node (RF) at (2 * 15, 0) {RF}; 
        \node (RN) at (1.5 * 15, 0.7) {RN};
        \node (RR) at (1 * 15, 0.7) {RR};

    \node (axisstart) at (0.5 * 15,-1) {};
    \node (axisend)   at (2.6 * 15,-1) {};
    \draw (axisstart) -- (axisend);
   \node (xaxislabel) at (2.6 * 15,-1.5) {\textsc{Sonority Rise}};

    \end{tikzpicture} % TODO: draw arrow and mark 1.0, 1.5, 2.0, 2.5
 \end{center}

Note that the clusters RF and RD fall between RT and NT on this scale.  Therefore, we cannot clearly separate the clusters that undergo epenthesis
from the ones that do not if we adopt {\sc Sonority Rise} as the relevant metric.
        
\bigskip

Carnie's (1994) analysis of these facts is in terms of a Minimal Distancing Constraint with various parameter settings, such as for which direction away from the nucleus it applies (to the right in the Irish case, to account for epenthesis applying to both codas and heterosyllabic clusters). His preliminary analysis has distance computed according to the following sonority scale:

\ex. \label{carniesonorityscale} \a. 1 - Voiceless stops
     \b. 2 - Voiced stops
     \b. 3 - Voiceless fricatives
     \b. 4 - Voiced fricatives
     \b. 5 - Sonorants

with a subsequent revision where sonority is computed in terms of the number of nodes that have to be specified to obtain the appropriate manner specification. This computation is somewhat complicated, so I will not reproduce it here. The result is that sonorant-fricative and sonorant-voiced stop clusters have a sonority distance of 0, since they are each specified for a single node in the feature geometry. Voiceless stops have no specified nodes, so sonorant-voiceless stop clusters have a sonority distance of 1, which is the minimal distance tolerated. 

A sonority scale that gives nasals and liquids identical sonority values seems unlikely given measurable differences in the phonetic correlates of sonority such as intensity between liquids and nasals \citep{parker.2002}. This is especially so when the obstruents are spread so widely on the sonority scale. 
I will therefore explore instead an alternative sonority distance-based analysis using Syllable Contact.

\bigskip

We quickly see that Syllable Contact on its own is insufficient as epenthesis in Irish also affects coda clusters. We will need to invoke Sonority Sequencing. 

%TODO : Syllable Contact and Sonority Sequencing should not be in smallcaps unless referred to as SyllCon and SonSeq.

Attempting to apply Syllable Contact to the Irish case, we immediately encounter problems. Firstly, it is difficult to see why markedness triggers epenthesis only in sonorant-initial clusters, as Syllable Contact predicts that obstruent-initial clusters should be even more marked.

Secondly, we would need to invoke a separate set of constraints on sonority sequencing in coda clusters, and assume that the ranking of the faithfulness constraint {\sc Dep} with respect to both sets of constraints is coincidentally identical.

Suppose we define a single markedness constraint that favours all falling sonority clusters regardless of their position in the syllable, i.e. combining Syllable Contact and Sonority Sequencing. We require a finer-grained metric to measure how marked a particular cluster is. An obvious course of action is to use \citet{gouskova.2002}'s {\sc *Dis} metric, extending its use from the heterosyllabic environment alone to all positions. 

Using the defined sonority scale, we expect the following markedness hierarchy of sonorant-initial clusters.

\begin{center}
  \begin{tikzpicture}[shorten >=1pt,->,scale=0.5]
     \tikzstyle{line} = [draw]%, -latex']

        \node (NT) at (2 * 5, 0) {NT}; 
        \node (ND) at (1.5 * 5, 0) {ND}; 
        \node (NF) at (1 * 5, 0) {NF};
        \node (NN) at (0 * 5, 0) {NN};
        \node (NR) at (-1 * 5, 0) {NR}; 

        \node (RT) at (3 * 5, 0) {RT};
        \node (RD) at (2.5 * 5, 0) {RD}; 
        \node (RF) at (2 * 5, 0.7) {RF}; 
        \node (RN) at (1 * 5, 0.7) {RN};
        \node (RR) at (0 * 5, 0.7) {RR};

    \node (axisstart) at (-1.5 * 5,-1) {};
    \node (axisend)   at (3.5 * 5,-1) {};
    \draw (axisstart) -- (axisend);
  \node (xaxislabel) at (3.5 * 5,-1.5) {\textsc{Dis}};

    \end{tikzpicture} 
 \end{center}

Even ignoring the problem of the obstruent-initial clusters, we see that sonority distance
alone does not get us the desired result. Instead, it incorrectly predicts that if NT and RT are not 
broken up by epenthesis, then RF and RD should not either.

\subsection{Interim summary}

To summarise the findings of the Irish case study: a sonority contour faithfulness account explains
the facts quite simply, because {\sc Sonority Angle} successfully achieves a linear separation of
the NT and RT clusters from the remaining sonorant-initial clusters, explaining why they fail to undergo
epenthesis. The {\sc Sonority Rise} metric does not have the same property. We have also seen that a markedness-based account would have to be much more complicated to explain the same facts.

\section{Case study: Chaha} \label{chaha}

Chaha is a Southern Semitic language spoken in central Ethiopia by about 440,000 people.  
As is common with Semitic languages, it has many underlying consonant clusters, some of which are resolved by epenthesis of the high central vowel [\textipa{1}].  Vowel epenthesis appears to preferentially target rising and level sonority clusters over falling ones.  The data in this section mostly comes from Rose 2000, who also provides an analysis in terms of Syllable Contact.

We will use the existing sonority scale given in \ref{standardsonorityscale}, with a small modification. Chaha has a bilabial approximant represented by [\textipa{B}] (more precisely, \textipa{\|`B}) \citep[15]{banksira.2000} that is not as sonorous as /r/, whose realisation is tap-like \citep[181]{taranto.2001}. The justification for /\textipa{B}/ being less sonorous than /r/ is largely empirical, based on the epenthetic behaviour of clusters containing this sonorant \citep[405]{rose.2000}, data for which will be presented later in this paper. I will specify its sonority as an intermediate 3.5, under the sonority of 4 for liquids in general.

\ex. Sonority scale to be used in Chaha

\begin{center}
\begin{tabular}{ccccccccccccc}
 V & $>$ & w j & $>$ & r & $>$ & \textipa{B} & $>$ & m n & $>$ & f s z x & $>$ & t t' k k' d g \\
 vowels & & glides & & liquids & &   & & nasals & & fricatives & & stops \\
    V   & &    G   & &    R    & & \textipa{B} & & N & & F & & T \\
    6   & &    5   & &    4    & &    3.5      & & 3 & & 2 & & 1 \\
\end{tabular}
\end{center}

\subsection{CC clusters}

\subsubsection{Data}

Underlying two-consonant clusters in Chaha are obligatorily epenthesised into if initial, never broken up by epenthesis in medial position, while final clusters are epenthesised into if they are of rising or level sonority, or the cluster consists of [r] followed by a nasal.

\ex. No onset clusters in Chaha
     \a. \textipa{k1r@tam} `he lifted up'
     
\ex. Tautomorphemic medial two-consonant clusters are tolerated no matter what their sonority profile
     \a. \textipa{s1rto} `cauterise ({\sc masc pl})!'
     \b. \textipa{n1k'mo} `pick ({\sc masc pl})!'
     \b. \textipa{at'met'} `solidified juice from [\textipa{@s@t}] plant'
     \z. \citep[(1a,c,d)]{rose.2000}

The presentation of the patterns of epenthesis in CC\# clusters is slightly complicated by variability between idiolects.  Epenthesis is possible in all level and rising sonority clusters.  
Among these clusters, the ones for which epenthesis is optionally unavailable are the obstruent-obstruent clusters.  Most falling sonority clusters do not undergo epenthesis,
except [r]-sonorant clusters, which can be broken up in certain idiolects. Voicing is irrelevant
to whether clusters undergo epenthesis.

The following table summarises the data.

\ex. Table showing epenthesis patterns for clusters of varying sonority patterns:

\begin{tabular}{c | c c c c c}
 \backslashbox{C1}{C2}      & Stop & Fricative & Nasal & Liquid \\ \hline
Stop   & \textbf{OPTIONAL}   & \textbf{OPTIONAL} & \textbf{\tickYes} & \textbf{\tickYes} \\
       & \textipa{n1g(1)d} & \textipa{d1g(1)s} & \textipa{n1k1m} & \textipa{g1d1r} \\
       & `touch!' & `give feast!' & `pick!' & `put to bed!' \\ \hline
Fric-  & \textbf{\tickNo} & \textbf{OPTIONAL} & \textbf{\tickYes} & \textbf{\tickYes} \\
ative  & \textipa{k1ft}   & \textipa{mes(1)x} &    & \textipa{s1f1r} \\
       & `open!'  & `chew!' & & `measure!' & \\ \hline
Nasal  & \textbf{\tickNo}     & \textbf{\tickNo}     & \textbf{\tickYes} & \textbf{\tickYes} \\
       & \textipa{g1nd} & \textipa{t1mx} & \textipa{g@n1m}   &  \\
       & `log (noun)'   & `scoop with finger!' & `take back loaned cow!' &  \\ \hline
Liquid & \textbf{\tickNo}     & \textbf{\tickNo} & \textbf{OPTIONAL}      & \textbf{\tickYes} \\
       & \textipa{f1rt} & \textipa{t1rx} & \textipa{k1r(1)m} & \textipa{s1B1r} \\
       & `divide in half!' & `make incision!' & `insult!'    & `break!' \\  \hline
\end{tabular}	
\noindent (Data assembled from \citet[404--7]{rose.2000})

\bigskip

The variability therefore lies in (1) whether epenthesis between obstruents is possible (between stops, between a stop and a fricative, and between two fricatives), and (2) whether [r]-sonorant can be broken up by epenthesis or not.

The variable setting of these two options corresponds to four attested idiolects:

  \ex. \label{idiolects} The four idiolects: % TODO: ref
       \a. \label{idiolecta} \tickYes OVO, \tickNo rVN \newline % \textsc{SonSeq} $\gg$ \textsc{SyllCon} $\gg$ \textsc{Dep-IO} $\gg$ \textsc{*R-Son} \newline
           [\textipa{k1rm}], [\textipa{s1B1r}], [\textipa{k1ft}], [\textipa{k1t1f}], [\textipa{s1g1d}]
       \b. \label{idiolectb} \tickYes OVO, \tickYes rVN \newline %\label{idiolectc} \textsc{SonSeq, *R-Son} $\gg$ \textsc{SyllCon} $\gg$ \textsc{Dep-IO} \newline
           [\textipa{k1r1m}], [\textipa{s1B1r}], [\textipa{k1ft}], [\textipa{k1t1f}], [\textipa{s1g1d}]
       \c. \label{idiolectc} \tickNo OVO, \tickNo rVN \newline %\label{idiolectb} \textsc{SyllCon} $\gg$ \textsc{Dep-IO} $\gg$ \textsc{SonSeq, *R-Son} \newline
           [\textipa{k1rm}], [\textipa{s1B1r}], [\textipa{k1ft}], [\textipa{k1tf}], [\textipa{s1gd}]
       \d. \label{idiolectd} \tickNo OVO, \tickYes rVN \newline %\label{idiolectd} \textsc{*R-Son} $\gg$ \textsc{SyllCon} $\gg$ \textsc{Dep-IO} $\gg$ \textsc{SonSeq} \newline
           [\textipa{k1r1m}], [\textipa{s1B1r}], [\textipa{k1ft}], [\textipa{k1tf}], [\textipa{s1gd}]


\subsubsection{Sonority Angle Analysis}

Under a sonority contour faithfulness analysis, epenthesis is triggered by a markedness constraint on the structural conditions of the cluster, not by the markedness of the underlying sonority contour profile itself. Undominated {\sc *ComplexOnset} accounts for the non-existence of initial clusters. {\sc *Coda} is ranked low, and therefore medial clusters are left intact.{\sc *ComplexCoda}, dominated by an {\sc Ident(SonAngle)} constraint, explains why some coda clusters are broken up by epenthesis while others are not.

In idiolect \ref{idiolecta}, all coda clusters with a {\sc Sonority Angle} smaller than 1.5 -- the rising and level sonority clusters -- are broken up by epenthesis, while those with a {\sc Sonority Angle} greater than 1.5 -- the falling sonority clusters -- are left intact. 

%TODO: diagram

The constraint ranking for this idiolect is thus:

\ex. {\sc Ident(SonAngle)}$<$1.5 $\gg$ {\sc *ComplexCoda}

\begin{center} \renewcommand*\arraystretch{1.2}
\scalebox{1}[1]{\begin{tabular}[t]{|rrl||c|c|} \hline 
\multicolumn{3}{|c||}{Input:~/\textipa{k1tf}/} & {\sc Ident(SonAngle)}$<$1.5 & {\sc *ComplexCoda} \\[0.5ex]
\hline \hline a. & & \textipa{k1tf} & & $\ast$! \\
\hline b. & \ding{43} & \textipa{k1t1f} & 0.59 & \\
\hline \end{tabular}} \renewcommand*\arraystretch{1} \end{center}

\begin{center} \renewcommand*\arraystretch{1.2}
\scalebox{1}[1]{\begin{tabular}[t]{|rrl||c|c|} \hline 
\multicolumn{3}{|c||}{Input:~/\textipa{k1rm}/} & {\sc Ident(SonAngle)}$<$1.5 & {\sc *ComplexCoda} \\[0.5ex]
\hline \hline c. &  \ding{43} & \textipa{k1rm} & & $\ast$ \\
\hline d. & & \textipa{k1r1m} & 1.89 $\ast$! & \\
\hline \end{tabular}} \renewcommand*\arraystretch{1} \end{center}

This approach does make one incorrect prediction: it predicts that /\textipa{Bm}/ should not undergo epenthesis under this ranking, since /\textipa{B}N/ has a {\sc Sonority Angle} of 1.65. But it always does:

\ex. /\textipa{s@B-m}/ $\rightarrow$ \textipa{s@B1m} `people-{\sc emph}'  % TODO: ref

The only clusters that give us evidence of this are /\textipa{Bm}/ clusters. These are not seen root-internally due to a high-ranking OCP effect. % TODO: ref: Greenberg cited in Rose
I suggest that there might be an undominated markedness constraint against this cluster -- either OCP(labial) or another constraint based on articulatory or perceptual difficulty -- that outranks all relevant faithfulness constraints.

\bigskip

Proceeding to idiolect \ref{idiolectb}, we find that [r]-sonorant clusters are allowed word-finally. Because {\sc Sonority Angle} permits a linear separation of the [r]-sonorant clusters from the remainder of the falling sonority clusters, this is straightforward to explain
by the movement of the cut-off line from 1.5 to 1.9.

% TODO: diagram

\ex. {\sc Ident(SonAngle)}$<$1.9 $\gg$ {\sc *ComplexCoda}

\begin{center} \renewcommand*\arraystretch{1.2}
\scalebox{1}[1]{\begin{tabular}[t]{|rrl||c|c|} \hline 
\multicolumn{3}{|c||}{Input:~/\textipa{k1rm}/} & {\sc Ident(SonAngle)}$<$1.5 & {\sc *ComplexCoda} \\[0.5ex]
\hline \hline a. &   & \textipa{k1rm} & & $\ast$! \\
\hline b. & \ding{43} & \textipa{k1r1m} & 1.89 & \\
\hline \end{tabular}} \renewcommand*\arraystretch{1} \end{center}

Idiolects \ref{idiolectc} and \ref{idiolectd} are identical to \ref{idiolecta} and \ref{idiolectb} respectively, except that obstruent-obstruent clusters do not undergo epenthesis under any circumstances. This can be achieved by ranking {\sc Dep}(+son)
above {\sc *ComplexCoda}.

\ex. Ranking for idiolect \ref{idiolectc}

\begin{center} \renewcommand*\arraystretch{1.2}
\scalebox{1}[1]{\begin{tabular}[t]{|rrl||c:c|c|} \hline 
\multicolumn{3}{|c||}{Input:~/\textipa{s1gd}/} & {\sc Dep}(+son) & {\sc Ident(SonAngle)}$<$1.5 & {\sc *ComplexCoda} \\[0.5ex]
\hline \hline a. & \ding{43} & \textipa{s1gd} & & & \cellcolor{lightgray}$\ast$ \\
\hline b. & & \textipa{s1g1d} & $\ast$! & 1.37 & \cellcolor{lightgray} \\
\hline \end{tabular}} \renewcommand*\arraystretch{1} \end{center}

To summarise, coda epenthesis behaviour in Chaha can be explained by the ranking of a {\sc Ident(SonAngle)}$<$n constraint above the markedness constraint triggering epenthesis, {\sc *ComplexCoda}. The evidence of the epenthetic behaviour of Chaha coda clusters accords with the {\sc Sonority Angle} prediction that RN should be the most likely to epenthesise out of the falling sonority clusters. It also demonstrates the action of a faithfulness constraint against introducing a sonorant region between two obstruents, {\sc Dep}(+son), as defined in $\S$\ref{depson}.

\ex. Rankings for the four idiolects:
     \a. {\sc Ident(SonAngle)}$<$1.5 $\gg$ {\sc *ComplexCoda} $\gg$ {\sc Dep}(+son)
     \b. {\sc Ident(SonAngle)}$<$1.9 $\gg$ {\sc *ComplexCoda} $\gg$ {\sc Dep}(+son)
     \c. {\sc Ident(SonAngle)}$<$1.5, {\sc Dep}(+son) $\gg$ {\sc *ComplexCoda}
     \c. {\sc Ident(SonAngle)}$<$1.9, {\sc Dep}(+son) $\gg$ {\sc *ComplexCoda}

\subsubsection{Comparison to other approaches}

{\sc Sonority Rise} can explain idiolects \ref{idiolecta} and \ref{idiolectc}, where the cut-off is drawn between rising and level sonority clusters on the one hand and falling sonority clusters on the other, it is less straightforward to explain the behaviour of idiolects \ref{idiolectb} and \ref{idiolectd}. This is because the cluster hierarchy predicted by {\sc Sonority Rise} does not permit a linear separation of [r]-sonorant clusters from the other falling sonority clusters.

% TODO: diagram

\bigskip

\citet{rose.2000} offers an explanation for these facts in terms of markedness of the sonority contours of the surface clusters. Medial clusters are subject to Syllable Contact, which bans rising sonority clusters across a syllable boundary, while coda clusters are subject to Sonority Sequencing.

The fact that medial clusters are never broken up by epenthesis even when they violate Syllable Contact is due to a constraint against the configuration [V.C\textipa{1}.CV], {\sc *MedialLight}. This constraint is unnecessary under my analysis as the only constraint that would militate against medial clusters, {\sc *Coda}, is not active.

In addition, Sonority Sequencing does not allow a divide of the [r]-sonorant clusters versus the remainder of the clusters, whether in Rose's categorical formulation of Sonority Sequencing or a family of {\sc *Dis} constraints.

% TODO: diagram

Therefore, Rose needs to invoke the following constraint to cause epenthesis in these clusters:

\ex. {\sc *R-Sonorant}: no [r]-sonorant sequences \citep[(41)]{rose.2000}

For those speakers that permit obstruent-obstruent clusters in defiance of Sonority Sequencing, Rose ranks Sonority Sequencing low, using the following constraint to trigger epenthesis in other clusters instead:

\ex. {\sc *C-Son\#}: a word-final sonorant consonant must be preceded by a vowel or [r].

Rose's approach thus requires the formulation of two constraints specifically targeting [r]. The advantage of doing so is that it avoids the difficulty of forbidding [\textipa{Bm}] clusters which we encountered above. On the other hand, the sonority contour faithfulness approach can get away without specifically mentioning [r] at all, by virtue of the {\sc Sonority Angle} hierarchy being able to group [r]-sonorant clusters together with the level and rising sonority clusters, at the expense of requiring a constraint against [\textipa{Bm}] clusters.

\subsection{CCC clusters}

\subsubsection{Data and Analysis}

Triconsonantal clusters are always broken up as [CC\textipa{1}C] or [C\textipa{1}CC]. The sources of CCC clusters in Chaha are the 3rd singular masculine conjugation of the jussive form of triliteral verbs, which have the underlying form /j\textipa{@}-\underline{CCC}-o/, and
the 3rd singular masculine conjugation of the passive/reflexive form of quadriliteral verbs,
which have the underlying form /j-\underline{t-CC}aCaC/.

As with coda clusters, there is considerable inter- and intra-speaker variability in epenthetic positioning between idiolects. The following data is from the main idiolect discussed in Section 4.3 of \citep{rose.2000}. Underlined forms have alternate forms in other idiolects.

In 


\section{Issues} \label{issues}

\section{Conclusion} \label{conclusion}

% TODO

\bibliographystyle{unified}
\bibliography{sonangle}


\end{document}
