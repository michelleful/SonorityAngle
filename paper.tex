\documentclass[12pt]{article}
\usepackage[margin=1in]{geometry}
\usepackage{fullpage}
\usepackage{mathpazo}
\usepackage{tipa}
\usepackage{setspace}
\usepackage{amsmath}
\usepackage{linguex}
\usepackage{tikz}
\usetikzlibrary{matrix,arrows}
\usepackage{slashbox}

\usepackage{amsfonts}
\newcommand{\tickYes}{\checkmark}
\usepackage{pifont}
\newcommand{\tickNo}{\hspace{1pt}\ding{55}}

\newcommand*\circled[1]{%
  \tikz[baseline=(C.base)]\node[draw,circle,inner sep=0.5pt](C) {#1};\!
}

\makeatletter
\newcommand\textsubscript[1]{\@textsubscript{\selectfont#1}}
\def\@textsubscript#1{{\m@th\ensuremath{_{\mbox{\fontsize\sf@size\z@#1}}}}}
\newcommand{\iep}{\textipa{1}}
\newcommand{\bet}{\textipa{B}}
\newcommand{\schwa}{\textipa{@}}

\title{The perceptual dimensions of \\ sonority-driven epenthesis}
\author{Michelle A. Fullwood}
\date{July 2013}
\begin{document}

\maketitle

\begin{abstract}
 
Vowel epenthesis often appears to preferentially target consonant clusters with rising sonority.
One explanation for this is perceptual faithfulness (Fleischhacker 2002, Steriade 2006): rising sonority clusters are more susceptible to epenthesis because the perceptual distance between the underlying /C\textsubscript{1}C\textsubscript{2}/ sequence and its correspondent output sequence [C\textsubscript{1}VC\textsubscript{2}] is small, thus incurring a smaller faithfulness cost.
This raises the question of how to compute the perceptual distance between two sonority contours /C\textsubscript{1}C\textsubscript{2}/ and [C\textsubscript{1}VC\textsubscript{2}] in terms of the sonority of C\textsubscript{1}, C\textsubscript{2} and V.  
In this paper, I propose that the appropriate metric is {\sc Sonority Angle}, the angle formed by the contours C\textsubscript{1}C\textsubscript{2} and C\textsubscript{1}V, and apply it in analyzing two case studies of sonority-driven epenthesis, Chaha and Irish.  A comparison is made to another possible metric,
{\sc Sonority Rise} (Flemming 2008), the ratio of the gradients of the two contours, as well as to Syllable Contact, which represents an alternative, markedness-based approach to the problem of sonority-driven epenthesis. 
\end{abstract}

\newpage
\tableofcontents
\newpage



\end{document}