\documentclass[12pt]{article}
\usepackage[margin=1in]{geometry}
\usepackage{fullpage}
\usepackage{natbib}
\usepackage{mathpazo}
\usepackage{tipa}
\usepackage{setspace}
\usepackage{amsmath}
\usepackage{linguex}
\usepackage{tikz}
\usetikzlibrary{matrix,arrows}
\usepackage{slashbox}

\usepackage{amsfonts}
\newcommand{\tickYes}{\checkmark}
\usepackage{pifont}
\newcommand{\tickNo}{\hspace{1pt}\ding{55}}

\newcommand*\circled[1]{%
  \tikz[baseline=(C.base)]\node[draw,circle,inner sep=0.5pt](C) {#1};\!
}

\makeatletter
\newcommand\textsubscript[1]{\@textsubscript{\selectfont#1}}
\def\@textsubscript#1{{\m@th\ensuremath{_{\mbox{\fontsize\sf@size\z@#1}}}}}
\newcommand{\iep}{\textipa{1}}
\newcommand{\bet}{\textipa{B}}
\newcommand{\schwa}{\textipa{@}}

\newsavebox{\sonorityanglehierarchycompressed}
\savebox{\sonorityanglehierarchycompressed}{
\begin{tikzpicture}[scale=0.8,shorten >=1pt,->]
  \tikzstyle{vertex}=[circle]
  \tikzstyle{point}=[circle,fill=black!25,minimum size=12pt,inner sep=2pt]
  \tikzstyle{line} = [draw, -latex']
  \node[vertex] (RT) at (2.4*8,1)   {RT};
  \node[vertex] (NT) at (2.4*8,0.5)  {NT};
  \node[vertex] (FT) at (2.1*8,1)  {FT};
  \node[vertex] (TT) at (1.4*8,1)     {TT};
  \node[vertex] (RF) at (2.2*8,1)     {RF};
  \node[vertex] (NF) at (2.0*8,1)  {NF};
  \node[vertex] (TF) at (0.6*8,1)   {TF};
  \node[vertex] (FF) at (1.3*8,1)   {FF};
  \node[vertex] (RN) at (1.9*8,1)   {RN};
  \node[vertex] (NN) at (1.2*8,1)   {NN};
  \node[vertex] (FN) at (0.54*8,0.5)  {FN};
  \node[vertex] (TN) at (0.3*8,1)   {TN};
  \node[vertex] (RR) at (1.1*8,1)  {RR};
  \node[vertex] (NR) at (0.46*8,1.0)  {NR};
  \node[vertex] (FR) at (0.2*8,1)   {FR};
  \node[vertex] (TR) at (0.1*8,1)   {TR};
  % axis
  \node[vertex] (axisstart) at (-0.23,-0.2) {};
  \node[vertex] (axisend)   at (2.7*8,-0.2) {};
  \draw (axisstart) -- (axisend);
  \draw (0.0*8, 0) -- (0.0*8, -0.4) -- cycle;
  \draw (1.0*8, 0) -- (1.0*8, -0.4) -- cycle;
  \draw (2.0*8, 0) -- (2.0*8, -0.4) -- cycle;
  \node[vertex] (0pointlabel) at (0.0,-0.7) {0};
  \node[vertex] (1pointlabel) at (1.0*8,-0.7) {1};
  \node[vertex] (2pointlabel) at (2.0*8,-0.7) {2};
  \node[vertex] (xaxislabel) at (2.7*8,-0.5) {\textsc{SonAngle}};
\end{tikzpicture}
}


\newsavebox{\sonorityanglehierarchycompressednumbers}
\savebox{\sonorityanglehierarchycompressednumbers}{
\begin{tikzpicture}[scale=0.8,shorten >=1pt,->]
  \tikzstyle{vertex}=[circle]
  \tikzstyle{point}=[circle,fill=black!25,minimum size=12pt,inner sep=2pt]
  \tikzstyle{line} = [draw, -latex']
  \node[vertex] (RT) at (2.4*8,1)   {RT};
  \node[vertex] (NT) at (2.4*8,0.5)  {NT};
  \node[vertex] (FT) at (2.1*8,1)  {FT};
  \node[vertex] (TT) at (1.4*8,1)     {TT};
  \node[vertex] (RF) at (2.2*8,1)     {RF};
  \node[vertex] (NF) at (2.0*8,1)  {NF};
  \node[vertex] (TF) at (0.6*8,1)   {TF};
  \node[vertex] (FF) at (1.3*8,1)   {FF};
  \node[vertex] (RN) at (1.9*8,1)   {RN};
  \node[vertex] (NN) at (1.2*8,1)   {NN};
  \node[vertex] (FN) at (0.54*8,0.5)  {FN};
  \node[vertex] (TN) at (0.3*8,1)   {TN};
  \node[vertex] (RR) at (1.1*8,1)  {RR};
  \node[vertex] (NR) at (0.46*8,1.0)  {NR};
  \node[vertex] (FR) at (0.2*8,1)   {FR};
  \node[vertex] (TR) at (0.1*8,1)   {TR};
  \node[vertex] (rising) at (0.35*8,2.5) {\underline{Rising sonority}};
  \node[vertex] (level) at (1.25*8,2.5) {\underline{Level sonority}};
  \node[vertex] (falling) at (2.1*8,2.5) {\underline{Falling sonority}};
  % axis
  \node[vertex] (axisstart) at (-0.23,-0.2) {};
  \node[vertex] (axisend)   at (2.7*8,-0.2) {};
  \draw (axisstart) -- (axisend);
  \draw (0.0*8, 0) -- (0.0*8, -0.4) -- cycle;
  \draw (1.0*8, 0) -- (1.0*8, -0.4) -- cycle;
  \draw (2.0*8, 0) -- (2.0*8, -0.4) -- cycle;
  \node[vertex] (0pointlabel) at (0.0,-0.7) {0};
  \node[vertex] (1pointlabel) at (1.0*8,-0.7) {1};
  \node[vertex] (2pointlabel) at (2.0*8,-0.7) {2};
  \node[vertex] (xaxislabel) at (2.7*8,-0.5) {\textsc{SonAngle}};
  \node (leastlikely) at (2.4*8,-1.5) {Least likely to epenthesise $\rightarrow$};
  \node (mostlikely) at (0.4*8,-1.5) {$\leftarrow$ Most likely to epenthesise};
  % numbers
  \node[font=\footnotesize,opacity=0.7] (RTno) at (2.4*8, 1.5) {2.36};
  \node[font=\footnotesize,opacity=0.7] (RFno) at (2.2*8, 1.5) {2.21};
  \node[font=\footnotesize,opacity=0.7] (FTno) at (2.1*8, 0.5) {2.11};
  \node[font=\footnotesize,opacity=0.7] (NFno) at (2.0*8, 1.5) {2.03};
  \node[font=\footnotesize,opacity=0.7] (RNno) at (1.9*8, 0.5) {1.89};

  \node[font=\footnotesize,opacity=0.7] (TTno) at (1.4*8, 1.5) {1.37};
  \node[font=\footnotesize,opacity=0.7] (FFno) at (1.3*8, 0.5) {1.33};
  \node[font=\footnotesize,opacity=0.7] (NNno) at (1.2*8, 1.5) {1.25};
  \node[font=\footnotesize,opacity=0.7] (RRno) at (1.1*8, 0.5) {1.11};

  \node[font=\footnotesize,opacity=0.7] (TFno) at (0.6*8, 1.5) {0.59};
  \node[font=\footnotesize,opacity=0.7] (FNno) at (0.54*8, 0) {0.54};
  \node[font=\footnotesize,opacity=0.7] (NRno) at (0.46*8, 1.5) {0.46};
  \node[font=\footnotesize,opacity=0.7] (TNno) at (0.3*8, 1.5) {0.27};
  \node[font=\footnotesize,opacity=0.7] (FRno) at (0.2*8, 0.5) {0.22};
  \node[font=\footnotesize,opacity=0.7] (TRno) at (0.1*8, 1.5) {0.12};

\end{tikzpicture}
}

\newsavebox{\sonorityrisehierarchycompressed}
\savebox{\sonorityrisehierarchycompressed}{
\begin{tikzpicture}[scale=0.8,shorten >=1pt,->]
  \tikzstyle{vertex}=[circle]
  \tikzstyle{point}=[circle,fill=black!25,minimum size=12pt,inner sep=2pt]
  \tikzstyle{line} = [draw, -latex']
  \node[vertex] (RT) at (2.5*8, 0.5)  {RT};
  \node[vertex] (RF) at (2*8,0.5)     {RF};
  \node[vertex] (NT) at (1.67*8,0.5)  {NT};
  \node[vertex] (RN) at (1.5*8,0.5)   {RN};
  \node[vertex] (NF) at (1.33*8,0)  {NF};
  \node[vertex] (FT) at (1.25*8,0.5)  {FT};

  \node[vertex] (TT) at (1*8,-0.5) {TT};
  \node[vertex] (FF) at (1*8,0)   {FF};
  \node[vertex] (NN) at (1*8,0.5)     {NN};
  \node[vertex] (RR) at (1*8,1.0)  {RR};

  \node[vertex] (TF) at (0.8*8,0)   {TF};
  \node[vertex] (FN) at (0.75*8,0.5)  {FN};
  \node[vertex] (NR) at (0.67*8,0)  {NR};
  \node[vertex] (TN) at (0.6*8,0.5)   {TN};
  \node[vertex] (FR) at (0.5*8,0.5)   {FR};
  \node[vertex] (TR) at (0.4*8,0.5)   {TR};
  % axis
  \node[vertex] (axisstart) at (-0.23,-1.0) {};
  \node[vertex] (axisend)   at (2.7*8,-1.0) {};
  \draw (axisstart) -- (axisend);
  \draw (0.0*8, -1.2) -- (0.0*8, -0.8) -- cycle;
  \draw (1.0*8, -1.2) -- (1.0*8, -0.8) -- cycle;
  \draw (2.0*8, -1.2) -- (2.0*8, -0.8) -- cycle;
  \node[vertex] (0pointlabel) at (0.0,-1.5) {0};
  \node[vertex] (1pointlabel) at (1.0*8,-1.5) {1};
  \node[vertex] (2pointlabel) at (2.0*8,-1.5) {2};
  \node[vertex] (xaxislabel) at (2.7*8,-1.5) {\textsc{SonRise}};

\end{tikzpicture}
}

\newsavebox{\sonorityrisehierarchycompressednumbers}
\savebox{\sonorityrisehierarchycompressednumbers}{
\begin{tikzpicture}[scale=0.8,shorten >=1pt,->]
  \tikzstyle{vertex}=[circle]
  \tikzstyle{point}=[circle,fill=black!25,minimum size=12pt,inner sep=2pt]
  \tikzstyle{line} = [draw, -latex']
  \node[vertex] (RT) at (2.5*8, 0.5)  {RT};
  \node[vertex] (RF) at (2*8,0.5)     {RF};
  \node[vertex] (NT) at (1.67*8,0.5)  {NT};
  \node[vertex] (RN) at (1.5*8,0.5)   {RN};
  \node[vertex] (NF) at (1.33*8,0)  {NF};
  \node[vertex] (FT) at (1.25*8,0.5)  {FT};

  \node[vertex] (TT) at (1*8,-0.5) {TT};
  \node[vertex] (FF) at (1*8,0)   {FF};
  \node[vertex] (NN) at (1*8,0.5)     {NN};
  \node[vertex] (RR) at (1*8,1.0)  {RR};

  \node[vertex] (TF) at (0.8*8,0)   {TF};
  \node[vertex] (FN) at (0.75*8,0.5)  {FN};
  \node[vertex] (NR) at (0.67*8,0)  {NR};
  \node[vertex] (TN) at (0.6*8,0.5)   {TN};
  \node[vertex] (FR) at (0.5*8,0.5)   {FR};
  \node[vertex] (TR) at (0.4*8,0.5)   {TR};
  \node[vertex] (rising) at (0.5*8,2) {\underline{Rising sonority}};
  \node[vertex] (level) at (1*8,2) {\underline{Level sonority}};
  \node[vertex] (falling) at (1.67*8,2) {\underline{Falling sonority}};
  % axis
  \node[vertex] (axisstart) at (-0.23,-1.0) {};
  \node[vertex] (axisend)   at (2.7*8,-1.0) {};
  \draw (axisstart) -- (axisend);
  \draw (0.0*8, -1.2) -- (0.0*8, -0.8) -- cycle;
  \draw (1.0*8, -1.2) -- (1.0*8, -0.8) -- cycle;
  \draw (2.0*8, -1.2) -- (2.0*8, -0.8) -- cycle;
  \node[vertex] (0pointlabel) at (0.0,-1.5) {0};
  \node[vertex] (1pointlabel) at (1.0*8,-1.5) {1};
  \node[vertex] (2pointlabel) at (2.0*8,-1.5) {2};
  \node[vertex] (xaxislabel) at (2.7*8,-1.5) {\textsc{SonRise}};

  % numbers
  \node[font=\footnotesize,opacity=0.7] (RTno) at (2.5*8, 1.0) {2.5};
  \node[font=\footnotesize,opacity=0.7] (RFno) at (2*8, 1.0) {2.0};
  \node[font=\footnotesize,opacity=0.7] (NTno) at (1.67*8, 1.0) {1.67};  
  \node[font=\footnotesize,opacity=0.7] (RNno) at (1.5*8, 1.0) {1.5};
  \node[font=\footnotesize,opacity=0.7] (NFno) at (1.33*8, -0.5) {1.33};
  \node[font=\footnotesize,opacity=0.7] (FTno) at (1.25*8, 1.0) {1.25};

  \node[font=\footnotesize,opacity=0.7] (levelno) at (1*8, 1.5) {1.0};

  \node[font=\footnotesize,opacity=0.7] (TFno) at (0.8*8, -0.5) {0.8};
  \node[font=\footnotesize,opacity=0.7] (FNno) at (0.75*8, 1.0) {0.75};
  \node[font=\footnotesize,opacity=0.7] (NRno) at (0.67*8, -0.5) {0.67};
  \node[font=\footnotesize,opacity=0.7] (TNno) at (0.6*8, 1.0) {0.6};
  \node[font=\footnotesize,opacity=0.7] (FRno) at (0.5*8, 1.0) {0.5};
  \node[font=\footnotesize,opacity=0.7] (TRno) at (0.4*8, 1.0) {0.4};

\end{tikzpicture}
}

\newsavebox{\syllablecontacthierarchy}
\savebox{\syllablecontacthierarchy}{
\begin{tikzpicture}[scale=0.8,shorten >=1pt,->]
  \tikzstyle{vertex}=[circle]
  \tikzstyle{point}=[circle,fill=black!25,minimum size=12pt,inner sep=2pt]
  \tikzstyle{line} = [draw, -latex']
  \node[vertex] (RT) at (3*3,2.0)   {RT};
  \node[vertex] (NT) at (2*3,1.5)  {NT};
  \node[vertex] (FT) at (1*3,1.0)  {FT};
  \node[vertex] (TT) at (0*3,0.5)     {TT};
  \node[vertex] (RF) at (2*3,2.0)     {RF};
  \node[vertex] (NF) at (1*3,1.5)  {NF};
  \node[vertex] (FF) at (0*3,1.0)   {FF};
  \node[vertex] (TF) at (-1*3,0.5)   {TF};
  \node[vertex] (RN) at (1*3,2.0)   {RN};
  \node[vertex] (NN) at (0*3,1.5)   {NN};
  \node[vertex] (FN) at (-1*3,1)  {FN};
  \node[vertex] (TN) at (-2*3,0.5)   {TN};
  \node[vertex] (RR) at (0*3,2.0)  {RR};
  \node[vertex] (NR) at (-1*3,1.5)  {NR};
  \node[vertex] (FR) at (-2*3,1.0)   {FR};
  \node[vertex] (TR) at (-3*3,0.5)   {TR};
  % headers
  \node[vertex] (rising) at (2*3,2.6) {\underline{Falling sonority}};
  \node[vertex] (level) at (0*3,2.6) {\underline{Level sonority}};
  \node[vertex] (falling) at (-2*3,2.6) {\underline{Rising sonority}};
  % axis
  \node[vertex] (axisstart) at (-3.5*3,-0) {};
  \node[vertex] (axisend)   at (3.5*3,-0) {};
  \draw (axisstart) -- (axisend);
  \draw (0, 0.2) -- (0, -0.2) -- cycle;
  \draw (1*3, 0.2) -- (1*3, -0.2) -- cycle;
  \draw (2*3, 0.2) -- (2*3, -0.2) -- cycle;
  \draw (3*3, 0.2) -- (3*3, -0.2) -- cycle;
  \draw (-1*3, 0.2) -- (-1*3, -0.2) -- cycle;
  \draw (-2*3, 0.2) -- (-2*3, -0.2) -- cycle;
  \draw (-3*3, 0.2) -- (-3*3, -0.2) -- cycle;
  \node[vertex] (0pointlabel) at (0.0,-0.5) {0};
  \node[vertex] (1pointlabel) at (1.0*3,-0.5) {1};
  \node[vertex] (2pointlabel) at (2.0*3,-0.5) {2};
  \node[vertex] (3pointlabel) at (3.0*3,-0.5) {3};
  \node[vertex] (-1pointlabel) at (-1.0*3,-0.5) {-1};
  \node[vertex] (-2pointlabel) at (-2.0*3,-0.5) {-2};
  \node[vertex] (-3pointlabel) at (-3.0*3,-0.5) {-3};
  \node[vertex] (xaxislabel) at (3.5*3,-0.5) {\textsc{Dis}};
\end{tikzpicture}
}





\title{The perceptual dimensions of \\ sonority-driven epenthesis}
\author{Michelle A. Fullwood}
\date{July 2013}
\begin{document}

\maketitle

\begin{abstract}
 
Vowel epenthesis often appears to preferentially target consonant clusters with rising sonority.
One explanation for this is perceptual faithfulness \citep{fleischhacker.2002,steriade.2006}: rising sonority clusters are more susceptible to epenthesis because the perceptual distance between the underlying /C\textsubscript{1}C\textsubscript{2}/ sequence and its correspondent output sequence [C\textsubscript{1}VC\textsubscript{2}] is small, thus incurring a smaller faithfulness cost.
This raises the question of how to compute the perceptual distance between two sonority contours /C\textsubscript{1}C\textsubscript{2}/ and [C\textsubscript{1}VC\textsubscript{2}] in terms of the sonority of C\textsubscript{1}, C\textsubscript{2} and V.  
In this paper, I propose that the appropriate metric is {\sc Sonority Angle}, the angle formed by the contours C\textsubscript{1}C\textsubscript{2} and C\textsubscript{1}V, and apply it in analyzing two case studies of sonority-driven epenthesis, Chaha and Irish.  A comparison is made to another possible metric,
{\sc Sonority Rise} \citep{flemming.2008}, the ratio of the gradients of the two contours, as well as to Syllable Contact, which represents an alternative, markedness-based approach to the problem of sonority-driven epenthesis. 
\end{abstract}

\newpage
\tableofcontents
\newpage

\section{Introduction}

Vowel epenthesis often appears to preferentially target consonant clusters with rising sonority.
There are two broad classes of explanation within Optimality Theory for such sonority-driven epenthesis.

One is faithfulness-based: the perceptual distance between the underlying /C\textsubscript{1}C\textsubscript{2}/ sequence and its correspondent
output sequence [C\textsubscript{1}VC\textsubscript{2}] is small when the cluster is of rising sonority.
Thus, epenthesis into such a sequence incurs a smaller faithfulness cost than epenthesis into a cluster of falling sonority. This is the basis of the analysis proposed by Fleischhacker (2002, 2005) to explain why rising sonority obstruent-sonorant clusters are more easily epenthesised in
to than falling sonority sibilant-stop clusters.

This faithfulness-based approach raises the question of how the perceptual distance between two sonority contours /C\textsubscript{1}C\textsubscript{2}/ and [C\textsubscript{1}VC\textsubscript{2}] should be computed in terms of the sonority of C\textsubscript{1}, C\textsubscript{2} and V.  
Fleischhacker's analysis rested on empirical determinations of sonority contour faithfulness, and did not attempt to determine such a relation.  

\citet{steriade.2006} proposed that input and output sonority contours should match in terms of whether they are rising or falling, and to what degree, but did not suggest a concrete mathematical relation.  \citet{flemming.2008} formalises Steriade's approach with the metric {\sc Sonority Rise}, the ratio of the gradients of the two contours.

\bigskip

In this paper, I suggest an alternative metric, {\sc Sonority Angle}, namely the magnitude of the angle made by the vectors C\textsubscript{1}-C\textsubscript{2} and C\textsubscript{1}-V, and explore the ramifications of this choice. 

{\sc Sonority Angle} makes the same broad predictions as {\sc Sonority Rise} -- that clusters of rising sonority, having a relatively small angle between the underlying sonority contour /C\textsubscript{1}-C\textsubscript{2}/ and the overt sonority contour [C\textsubscript{1}-V], are perceptually more similar to their epenthetic output, and therefore more likely to undergo epenthesis, than clusters of falling sonority. Crucially, however, the exact hierarchy of susceptibility of individual clusters to epenthesis is predicted to be different.

I take two instances where the predictions of {\sc Sonority Angle} and {\sc Sonority Rise} differ and illustrate with case studies of sonority-driven epenthesis in two different languages, namely Chaha and Irish, that the predictions of {\sc Sonority Angle} are more in line with the data than those of {\sc Sonority Rise}.

\bigskip

The other broad class of explanation for sonority-driven epenthesis is markedness-based.  Syllable Contact \citep{murray.vennemann.1983} holds that across a syllable boundary, falling sonority clusters are more harmonic than rising sonority ones.  Hence, rising sonority clusters are preferentially broken up by epenthesis.

Syllable Contact forms the basis for the main existing analysis of Chaha epenthesis by \citet{rose.2000}.  I show that the faithfulness-based analysis, powered by the metric of {\sc Sonority Angle}, is more economical.  In the case of Irish, Syllable Contact makes incorrect predictions regarding the data.

\bigskip

The layout of this paper is as follows.  Section \ref{theoreticalmachinery} lays out the theoretical background for the sonority contour faithfulness approach to 
sonority-driven epenthesis. I introduce the proposed {\sc Sonority Angle} metric as well as the competing {\sc Sonority Rise} metric \citep{flemming.2008}, 
then lay out the alternative markedness-based approach to sonority-driven epenthesis, namely {\sc Syllable Contact}.

Section \ref{irish} consists of a case study of sonority-driven epenthesis in Irish.  I show that the data are in line with the predictions of {\sc Sonority Angle} and not {\sc Sonority Rise}, while a Syllable Contact-based analysis would have to be very complicated to explain the same facts.

Section \ref{chaha} is a case study of epenthesis positioning in Chaha.  I detail the facts of epenthesis positioning in Chaha, based on the data given in \citet{rose.2000},
and show that the sonority contour faithfulness approach explains these facts, with {\sc Sonority Angle} as the metric for comparing sonority contours.  I compare it to {\sc Sonority Rise} and show that the former is the more successful analysis, and that overall, the approach just outlined is more economical than the 
Syllable Contact-based approach of \citet{rose.2000}.

Section \ref{issues} looks at various issues regarding {\sc Sonority Angle}, such as its robustness.
Section \ref{conclusion} concludes.

\section{Theoretical background} \label{theoreticalmachinery}

This paper assumes as its basis the P-map hypothesis \citep{steriade.2001}, which states that the perceptual distance between underlying representations and potential surface forms projects a fixed ranking of faithfulness constraints.

In order to determine what faithfulness constraints exist in {\sc Con} and what their rankings should be, therefore, we need to know the metrics of perceptual distance that are relevant to each change. In the case of vowel epenthesis, the perceptual distance to be measured is between two sonority contours, /C1-C2/ and [C\textsubscript{1}-V-C\textsubscript{2}].

\subsection{Sonority Angle}

The observation with which we started was that the more steeply rising the sonority profile of a consonant cluster, the more likely the cluster to undergo epenthesis. Thus the absolute difference in sonority between C\textsubscript{1} and C\textsubscript{2} must be factored into the metric. To this, I add the claim that the more sonorous C\textsubscript{1}, the more likely the cluster is to undergo epenthesis.

These two factors are neatly captured by the metric {\sc Sonority Angle}, which is defined as the angle between the underlying C\textsubscript{1}C\textsubscript{2} sonority contour and the surface C\textsubscript{1}V contour:

\ex. \label{sonangle_picture} \begin{tikzpicture} [shorten >=1pt,scale=0.33]
                      \draw [<-] (-3,6.5) -- (-3, 0.5) ; % axis
                      \node at (-3, 7.0) {Sonority}; % axis label
                      \draw (5,1) -- (0,3) -- (5,6) ; % rising sonority
                      \node at (1.8, 3.2) {$\theta$}; % theta label for rising sonority
                      \node[left] at (0,3) {C$_1$}; 
                      \node[right] at (5,1) {C$_2$};
                      \node[right] at (5,6) {V};
    \coordinate (A) at (0,3);
    \coordinate (B) at (5,1);
    \coordinate (C) at (5,6);
\end{tikzpicture} 

Assuming that the horizontal distance is 1 unit, we can compute the magnitude of this angle analytically with the following formula:

\ex. \label{sonangle_formula} Formula: \textsc{SonAngle} = $arctan(V-C_1) - arctan(C_2-C_1)$

Let us verify that {\sc Sonority Angle} does indeed reflect the two generalisations we wish to make:
first, that the smaller the sonority distance between C\textsubscript{1} and C\textsubscript{2},
the smaller the sonority angle.
Imagine fixing C\textsubscript{1} as in \ref{sonangle_picture} and raising the sonority of C\textsubscript{2}. Intuitively, this decreases the {\sc Sonority Angle}, comparing the two below.

\ex. \begin{tikzpicture} [shorten >=1pt,scale=0.33]
                      \draw [<-] (-3,6.5) -- (-3, 0.5) ; % axis
                      \node at (-3, 7.0) {Sonority}; % axis label
					 % left diagram
                      \draw (5,1) -- (0,3) -- (5,6) ; % rising sonority
                      \node at (1.8, 3.2) {$\theta$}; % theta label for rising sonority
                      \node[left] at  (0,3) {C$_1$}; 
                      \node[right] at (5,1) {C$_2$};
                      \node[right] at (5,6) {V};
					  % right diagram
                      \draw (20,4) -- (15,3) -- (20,6) ; % falling sonority
                      \node at (17.5, 4.0) {$\theta$}; % theta label for falling sonority
                      \node[left] at (15,3) {C$_1$}; 
                      \node[right] at (20,4) {C$_2$};
                      \node[right] at (20,6) {V};
\end{tikzpicture} 

The dependence of {\sc Sonority Angle} on this distance can also be seen in the second term in \ref{sonangle_formula}.

The second generalisation is that the more sonorous the C\textsubscript{1}, the more likely
epenthesis is to occur. This time, fix C\textsubscript{2} and lower the sonority of C\textsubscript{1}.

\ex. \begin{tikzpicture} [shorten >=1pt,scale=0.33]
                      \draw [<-] (-3,6.5) -- (-3, 0.5) ; % axis
                      \node at (-3, 7.0) {Sonority}; % axis label
					  % left diagram
                      \draw (5,1) -- (0,3) -- (5,6) ; % rising sonority
                      \node at (1.8, 3.2) {$\theta$}; % theta label for rising sonority
                      \node[left] at  (0,3) {C$_1$}; 
                      \node[right] at (5,1) {C$_2$};
                      \node[right] at (5,6) {V};
					  % right diagram
                      \draw (20,1) -- (15,1) -- (20,6) ; % falling sonority
                      \node at (17, 1.9) {$\theta$}; % theta label for falling sonority
                      \node[left] at (15,1) {C$_1$}; 
                      \node[right] at (20,1) {C$_2$};
                      \node[right] at (20,6) {V};
\end{tikzpicture} 

While the difference is less clear visually, the second angle is larger. The first term in the formula confirms the relation between the sonority of C\textsubscript{1} in terms of its closeness to V, and {\sc Sonority Angle} as a whole.

Given a sonority scale where classes of consonants are mapped to a numerical sonority, we can now calculate
the {\sc Sonority Angle} for any cluster, which can be thought of as the faithfulness cost of epenthesising
between the two consonants.  Examples of the calculation are given below.

In this paper, I adopt (with, later, minor modifications) the following standard scale:

\ex. \label{standardsonorityscale} 
      \begin{tabular}{cccccc}
         T & F & N & R & G & V \\
         stop & fricative & nasal & liquid & glide & vowel \\
         1 & 2 & 3 & 4 & 5 & 6 \\
      \end{tabular}

The {\sc Sonority Angles} for NT, TT and TN are calculated as in the following examples.

\ex. \a. \textsc{SonAngle}(NT) =  $arctan(6-3) - arctan(1-3)$ = 2.35
     \b. \textsc{SonAngle}(TT) =  $arctan(6-1) - arctan(1-1)$ = 1.37
     \c. \textsc{SonAngle}(TN) =  $arctan(6-1) - arctan(3-1)$ = 0.27

The larger the {\sc Sonority Angle}, the larger the faithfulness cost. We therefore expect it to be hardest to epenthesise into NT out of these three clusters, and easiest to epenthesise into TN.

We formalise the idea of the faithfulness cost by defining a family of \textsc{Ident} constraints that penalise outputs that incur faithfulness costs of greater than a certain $n$.

\ex. \textsc{Ident(SonAngle)}$<n$: Assign a violation mark if the consonants in two strings C\textsubscript{1}C\textsubscript{2} and C\textsubscript{1}VC\textsubscript{2} 
stand in correspondence, and the sonority angle between C\textsubscript{1}C\textsubscript{2} and C\textsubscript{1}V is greater than $n$. % TODO: change to it being a property of C1?

These faithfulness constraints have a universal ranking, with the least stringent the highest-ranked.

   \ex. ... \newline
            $\gg$ \textsc{Ident(SonAngle)}$<$1.5 \newline
            $\gg$ \textsc{Ident(SonAngle)}$<$1.0 \newline
            $\gg$ \textsc{Ident(SonAngle)}$<$0.5 \newline
            $\gg$ ...

The resulting hierarchy of clusters, ranked according to their resistance to epenthesis as defined
by their {\sc Sonority Angle}, is as follows.

\ex. {\sc Sonority Angle} hierarchy

\vspace{-3em}
\noindent \resizebox{\linewidth}{!}{\usebox{\sonorityanglehierarchycompressed}}

Notice that out of the falling sonority clusters, those that decrease in sonority by a single step
-- namely RN, NF and FT -- have smaller {\sc Sonority Angles} than the ones that have a greater fall
in sonority. Furthermore, out of these three clusters, theone with the most sonorous C\textsubscript{1}, RN, has the smallest {\sc Sonority Angle}. We thus predict that out of the falling sonority clusters,
RN is the most likely to be broken up by epenthesis. The case study on Chaha will demonstrate that this
is the case.

Similarly, between the clusters that fall in sonority by two steps -- RF and NT -- we expect
NT to be less likely to undergo epenthesis, since N is less sonorous than R. We expect NT and RT
to be the clusters most resistant to epenthesis out of all the clusters. This will be crucial 
to our analysis of Irish sonority-driven epenthesis.

\subsection{Sonority Rise}

I will contrast {\sc Sonority Angle} with an alternative metric of sonority contour faithfulness proposed by \citet{flemming.2008}, {\sc Sonority Rise}, which takes the ratio of the underlying sonority contour /C\textsubscript{1}C\textsubscript{2}/ and the surface contour [C\textsubscript{1}V]. 

   \ex. \textsc{Sonority Rise} = $1-\frac{C_2-C_1}{V-C_1}$
   
The following sample calculations illustrate how \textsc{SonRise} distance is computed:

\ex.    \begin{center}
    \begin{tabular}{c c | c c | c}
    C\textsubscript{1}C\textsubscript{2}   & Rise & C\textsubscript{1}V & Rise & \textsc{SonRise} Distance\\ \hline
    NT &  -2  &  N\textipa{1}T & 3 & $1-\frac{1-3}{6-3}=1.7$ \\
    TT &  0   &  T\textipa{1}T & 5 & $1-\frac{1-1}{6-1}=1.0$ \\
    TN &  2   &  T{\textipa{1}}N & 5 & $1-\frac{3-1}{6-1}=0.6$ \\ 
    \end{tabular}
    \end{center}

As with {\sc Sonority Angle}, this gives rise to a family of constraints {\sc Ident(SonRise)}$<n$, defined similarly to {\sc Ident(SonAngle)}$<n$.

{\sc Sonority Rise} also gives rise to a hierarchy of susceptbility of clusters.

\ex. {\sc Sonority Rise} hierarchy

\vspace{-3em}
\noindent \resizebox{\linewidth}{!}{\usebox{\sonorityrisehierarchycompressed}}

Though in many ways similar to the {\sc Sonority Angle} hierarchy, it makes several crucially different predictions. For example, it does not share the prediction of {\sc Sonority Angle} that RN is the most likely of the falling sonority clusters to epenthesise -- rather, if RN undergoes epenthesis then we expect FT and NF to do the same. Furthermore, it predicts that if NT and RT fail to undergo epenthesis due to the high faithfulness cost of interrupting these clusters, then RF should also fail to undergo epenthesis. Both of these predictions are contrary to the evidence of Irish and Chaha.

\subsection{Another dimension of sonority contour faithfulness}

Note that I do not claim that sonority contour faithfulness is the only dimension of perceptual similarity relevant to vowel epenthesis. For instance, I will adopt \citet{flemming.2008}'s suggestion that a further dimension of perceptual similarity restricts epenthesis to sites adjacent to a sonorant, based on patterns of epenthesis in languages such as Montana Salish, where epenthesis is disallowed between two obstruents. 

I will formalise this restriction with the constraint {\sc Dep(+sonorant)}. The epenthesis of a vowel between two obstruents, which bear the feature [$-$sonorant], necessitates the insertion of a new [+sonorant] feature, whereas when a vowel is epenthesised adjacent to a sonorant, the sonorant's [+sonorant] feature spreads and is shared between the vowel and the sonorant.

This constraint will be used in the analysis of Chaha complex coda epenthesis, which in some idiolects
cannot occur between obstruents, while in others, it can. My proposal will be that in the former, {\sc Dep(+sonorant)} is higher-ranked, thus blocking epenthesis.

\subsection{Syllable Contact} 

The alternative markedness-based approach to sonority-driven epenthesis that I will explore in this paper is Syllable Contact, which was first stated as the Syllable Contact Law by \citet{murray.vennemann.1983}:

\ex. ``The preference for a syllabic structure $A$\$$B$, where $A$ and $B$ are marginal segments and $a$ and $b$ are the Consonantal Strength
values of $A$ and $B$ respectively, increases with the value of $b$ minus $a$'' \citep{murray.vennemann.1983}

\citet{rose.2000} reformulates this as a violable, but categorical, constraint within the context of Optimality Theory and uses it in an analysis of Chaha sonority-driven epenthesis:

\ex.  \textsc{SyllCon}: The first segment of the onset of a syllable must be lower in sonority than the last segment in the immediately preceding syllable.

More recently, Syllable Contact has been recast as a gradient family of constraints, for example by \citet{gouskova.2002, gouskova.2004}.  She defines the distance {\sc Dis} between two consonants in a syllable contact situation as the sonority of the second minus the sonority of the first.  For example, [t.s] = +1 if [s] and [t] are only one step apart on a sonority scale.  She then defines the following family of constraints:

\ex. {\sc *Dis-}$n$: Assign a violation mark if the {\sc Dis} between two adjacent heterosyllabic consonants is $n$ (adapted from \citep{gouskova.2002})

This gives rise to a universal hierarchy:

\ex. {\sc *Dis+7} $\gg$ {\sc *Dis+6} $\gg$ ... {\sc *Dis+0} $\gg$ {\sc *Dis-1} $\gg$ ... {\sc *Dis-7}.

Thus heterosyllabic clusters that rise sharply in sonority -- that have a high {\sc Dis} -- are more marked than more falling clusters.

The predicted hierarchy of susceptibility to epenthesis of the clusters is as follows:

\ex. Syllable Contact hierarchy (based on {\sc *Dis}):

\vspace{-3em}
\noindent \resizebox{\linewidth}{!}{\usebox{\syllablecontacthierarchy}}

\bigskip

Syllable Contact applies only across syllable boundaries.  Therefore, when sonority-driven epenthesis occurs in onsets or codas, other sonority-based markedness constraints must be employed.  One such is {\sc Sonority Sequencing} \citep{selkirk.1984}, which states that sonority must be strictly increasing in the onset and strictly decreasing in the coda. \citep{rose.2000} uses two variants of this constraint, one strict and one non-strict, in her analysis of Chaha.

\section{Case study: Irish} \label{irish}

Irish displays an unusual process of epenthesis that targets consonant clusters 
whose first member is a sonorant \citep{carnie.1994, ni.chiosain.1999}, with the 
exception of sonorant-voiceless stop clusters, which never undergo epenthesis. 
I will show in this section that a {\sc Sonority Angle}-based faithfulness
neatly captures these facts.



\section{Case study: Chaha} \label{chaha}

\section{Issues} \label{issues}

\section{Conclusion} \label{conclusion}

\bibliographystyle{unified}
\bibliography{sonangle}


\end{document}