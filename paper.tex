\documentclass[12pt]{article}
\usepackage[margin=1in]{geometry}
\usepackage{fullpage}
\usepackage{natbib}
\usepackage{mathpazo}
\usepackage{tipa}
\usepackage{setspace}
\usepackage{amsmath}
\usepackage{linguex}
\usepackage{tikz}
\usetikzlibrary{matrix,arrows}
\usepackage{slashbox}

\usepackage{amsfonts}
\newcommand{\tickYes}{\checkmark}
\usepackage{pifont}
\newcommand{\tickNo}{\hspace{1pt}\ding{55}}

\newcommand*\circled[1]{%
  \tikz[baseline=(C.base)]\node[draw,circle,inner sep=0.5pt](C) {#1};\!
}

\makeatletter
\newcommand\textsubscript[1]{\@textsubscript{\selectfont#1}}
\def\@textsubscript#1{{\m@th\ensuremath{_{\mbox{\fontsize\sf@size\z@#1}}}}}
\newcommand{\iep}{\textipa{1}}
\newcommand{\bet}{\textipa{B}}
\newcommand{\schwa}{\textipa{@}}

\title{The perceptual dimensions of \\ sonority-driven epenthesis}
\author{Michelle A. Fullwood}
\date{July 2013}
\begin{document}

\maketitle

\begin{abstract}
 
Vowel epenthesis often appears to preferentially target consonant clusters with rising sonority.
One explanation for this is perceptual faithfulness (\cite{fleischhacker.2002,steriade.2006}): rising sonority clusters are more susceptible to epenthesis because the perceptual distance between the underlying /C\textsubscript{1}C\textsubscript{2}/ sequence and its correspondent output sequence [C\textsubscript{1}VC\textsubscript{2}] is small, thus incurring a smaller faithfulness cost.
This raises the question of how to compute the perceptual distance between two sonority contours /C\textsubscript{1}C\textsubscript{2}/ and [C\textsubscript{1}VC\textsubscript{2}] in terms of the sonority of C\textsubscript{1}, C\textsubscript{2} and V.  
In this paper, I propose that the appropriate metric is {\sc Sonority Angle}, the angle formed by the contours C\textsubscript{1}C\textsubscript{2} and C\textsubscript{1}V, and apply it in analyzing two case studies of sonority-driven epenthesis, Chaha and Irish.  A comparison is made to another possible metric,
{\sc Sonority Rise} (\cite{flemming.2008}), the ratio of the gradients of the two contours, as well as to Syllable Contact, which represents an alternative, markedness-based approach to the problem of sonority-driven epenthesis. 
\end{abstract}

\newpage
\tableofcontents
\newpage

\section{Introduction}

Vowel epenthesis often appears to preferentially target consonant clusters with rising sonority.
There are two broad classes of explanation within Optimality Theory for such sonority-driven epenthesis.

One is faithfulness-based: the perceptual distance between the underlying /C\textsubscript{1}C\textsubscript{2}/ sequence and its correspondent
output sequence [C\textsubscript{1}VC\textsubscript{2}] is small when the cluster is of rising sonority.
Thus, epenthesis into such a sequence incurs a smaller faithfulness cost than epenthesis into a cluster of falling sonority. This is the basis of the analysis proposed by Fleischhacker (2002, 2005) to explain why rising sonority obstruent-sonorant clusters are more easily epenthesised in
to than falling sonority sibilant-stop clusters.

This faithfulness-based approach raises the question of how the perceptual distance between two sonority contours /C\textsubscript{1}C\textsubscript{2}/ and [C\textsubscript{1}VC\textsubscript{2}] should be computed in terms of the sonority of C\textsubscript{1}, C\textsubscript{2} and V.  
Fleischhacker's analysis rested on empirical determinations of sonority contour faithfulness, and did not attempt to determine such a relation.  

\citet{steriade.2006} proposed that input and output sonority contours should match in terms of whether they are rising or falling, and to what degree, but did not suggest a concrete mathematical relation.  \citet{flemming.2008} formalizes Steriade's approach with the metric {\sc Sonority Rise}, the ratio of the gradients of the two contours.

\bigskip

In this paper, I suggest an alternative metric, {\sc Sonority Angle}, namely the magnitude of the angle made by the vectors C\textsubscript{1}-C\textsubscript{2} and C\textsubscript{1}-V, and explore the ramifications of this choice. 

{\sc Sonority Angle} makes the same broad predictions as {\sc Sonority Rise} -- that clusters of rising sonority, having a relatively small angle between the underlying sonority contour /C\textsubscript{1}-C\textsubscript{2}/ and the overt sonority contour [C\textsubscript{1}-V], are perceptually more similar to their epenthetic output, and therefore more likely to undergo epenthesis, than clusters of falling sonority. Crucially, however, the exact hierarchy of susceptibility of clusters to epenthesis is predicted to be different.

I take two instances where the predictions of {\sc Sonority Angle} and {\sc Sonority Rise} differ and illustrate with case studies of sonority-driven epenthesis in two different languages, namely Chaha and Irish, that the predictions of {\sc Sonority Angle} are more in line with the data than those of {\sc Sonority Rise}.

\bigskip

The other broad class of explanation for sonority-driven epenthesis is markedness-based.  Syllable Contact \cite{murray.vennemann.1983} holds that across a syllable boundary, falling sonority clusters are more harmonic than rising sonority ones.  Hence, rising sonority clusters are preferentially broken up by epenthesis.

Syllable Contact forms the basis for the main existing analysis of Chaha epenthesis by \citet{rose.2000}.  I show that the faithfulness-based analysis, powered by the metric of {\sc Sonority Angle}, is more economical.  In the case of Irish, Syllable Contact makes incorrect predictions regarding the data.

\bigskip

The layout of this paper is as follows.  Section \ref{theoreticalmachinery} lays out the theoretical background for the sonority contour faithfulness approach to 
sonority-driven epenthesis. I introduce the proposed {\sc Sonority Angle} metric as well as the competing {\sc Sonority Rise} metric \cite{flemming.2008}, 
then lay out the alternative markedness-based approach to sonority-driven epenthesis, namely {\sc Syllable Contact}.

Section \ref{irish} consists of a case study of sonority-driven epenthesis in Irish.  Similarly to Irish, I show that the data are in line with the predictions of {\sc Sonority Angle} and not {\sc Sonority Rise}, while a Syllable Contact-based analysis would have to be very complicated to explain the same facts.

Section \ref{chaha} is a major case study of epenthesis positioning in Chaha.  I detail the facts of epenthesis positioning in Chaha, based on the data given in \citet{rose.2000},
and show that the sonority contour faithfulness approach explains these facts, with {\sc Sonority Angle} as the metric for comparing sonority contours.  I compare it to {\sc Sonority Rise} and show that the former is the more successful analysis, and that overall, the approach just outlined is more economical than the 
Syllable Contact-based approach of \citet{rose.2000}.

Section \ref{issues} looks at various issues regarding {\sc Sonority Angle}, such as its robustness.
Section \ref{conclusion} concludes.

\section{Theoretical background} \label{theoreticalmachinery}

This paper assumes as its basis the P-map hypothesis \cite{steriade.2001}, which states that the perceptual distance between underlying representations and potential surface forms projects a fixed ranking of faithfulness constraints.

In order to determine what faithfulness constraints exist in {\sc Con} and what their rankings should be, therefore, we need to know the metrics of perceptual distance that are relevant to each change. In the case of vowel epenthesis, the perceptual distance to be measured is between two sonority contours, /C1-C2/ and [C\textsubscript{1}-V-C\textsubscript{2}].

\subsection{Sonority Angle}

The observation with which we started was that the more steeply rising the sonority profile of a consonant cluster, the more likely the cluster to undergo epenthesis. Thus the absolute difference in sonority between C\textsubscript{1} and C\textsubscript{2} must be factored into the metric. To this, I add the claim that the more sonorous C\textsubscript{1}, the more likely the cluster is to undergo epenthesis.

These two factors are neatly captured by the metric {\sc Sonority Angle}, which is defined as the angle between the underlying C\textsubscript{1}C\textsubscript{2} sonority contour and the surface C\textsubscript{1}V contour:

\ex. \label{sonangle_picture} \begin{tikzpicture} [shorten >=1pt,scale=0.33]
                      \draw [<-] (-3,6.5) -- (-3, 0.5) ; % axis
                      \node at (-3, 7.0) {Sonority}; % axis label
                      \draw (5,1) -- (0,3) -- (5,6) ; % rising sonority
                      \node at (1.8, 3.2) {$\theta$}; % theta label for rising sonority
                      \node[left] at (0,3) {C$_1$}; 
                      \node[right] at (5,1) {C$_2$};
                      \node[right] at (5,6) {V};
    \coordinate (A) at (0,3);
    \coordinate (B) at (5,1);
    \coordinate (C) at (5,6);
\end{tikzpicture} 

Assuming that the horizontal distance is 1 unit, we can compute the magnitude of this angle analytically with the following formula:

\ex. \label{sonangle_formula} Formula: \textsc{SonAngle} = $arctan(V-C_1) - arctan(C_2-C_1)$

Let us verify that {\sc Sonority Angle} does indeed reflect the two generalisations we wish to make:
first, that the smaller the sonority distance between C\textsubscript{1} and C\textsubscript{2},
the smaller the sonority angle.
Imagine fixing C\textsubscript{1} as in \ref{sonangle_picture} and raising the sonority of C\textsubscript{2}. Intuitively, this decreases the {\sc Sonority Angle}, comparing the two below.

\ex. \begin{tikzpicture} [shorten >=1pt,scale=0.33]
                      \draw [<-] (-3,6.5) -- (-3, 0.5) ; % axis
                      \node at (-3, 7.0) {Sonority}; % axis label
					 % left diagram
                      \draw (5,1) -- (0,3) -- (5,6) ; % rising sonority
                      \node at (1.8, 3.2) {$\theta$}; % theta label for rising sonority
                      \node[left] at  (0,3) {C$_1$}; 
                      \node[right] at (5,1) {C$_2$};
                      \node[right] at (5,6) {V};
					  % right diagram
                      \draw (20,4) -- (15,3) -- (20,6) ; % falling sonority
                      \node at (17.5, 4.0) {$\theta$}; % theta label for falling sonority
                      \node[left] at (15,3) {C$_1$}; 
                      \node[right] at (20,4) {C$_2$};
                      \node[right] at (20,6) {V};
\end{tikzpicture} 

The dependence of {\sc Sonority Angle} on this distance can also be seen in the second term in \ref{sonangle_formula}.

The second generalisation is that the more sonorous the C\textsubscript{1}, the more likely
epenthesis is to occur. This time, fix C\textsubscript{2} and lower the sonority of C\textsubscript{1}.

\ex. \begin{tikzpicture} [shorten >=1pt,scale=0.33]
                      \draw [<-] (-3,6.5) -- (-3, 0.5) ; % axis
                      \node at (-3, 7.0) {Sonority}; % axis label
					  % left diagram
                      \draw (5,1) -- (0,3) -- (5,6) ; % rising sonority
                      \node at (1.8, 3.2) {$\theta$}; % theta label for rising sonority
                      \node[left] at  (0,3) {C$_1$}; 
                      \node[right] at (5,1) {C$_2$};
                      \node[right] at (5,6) {V};
					  % right diagram
                      \draw (20,1) -- (15,1) -- (20,6) ; % falling sonority
                      \node at (17, 1.9) {$\theta$}; % theta label for falling sonority
                      \node[left] at (15,1) {C$_1$}; 
                      \node[right] at (20,1) {C$_2$};
                      \node[right] at (20,6) {V};
\end{tikzpicture} 

While the difference is less clear visually, the second angle is larger. The first term in the formula confirms the relation between the sonority of C\textsubscript{1} in terms of its closeness to V, and {\sc Sonority Angle} as a whole.

Given a sonority scale where classes of consonants are mapped to a numerical sonority, we can now calculate
the {\sc Sonority Angle} for any cluster, which can be thought of as the faithfulness cost of epenthesising
between the two consonants.  Examples of the calculation are given below.

In this paper, I adopt (with, later, minor modifications) the following standard scale:

\ex. \label{standardsonorityscale} 
      \begin{tabular}{cccccc}
         T & F & N & R & G & V \\
         stop & fricative & nasal & liquid & glide & vowel \\
         1 & 2 & 3 & 4 & 5 & 6 \\
      \end{tabular}

The {\sc Sonority Angles} for NT, TT and TN are calculated as follows.

\ex. \a. \textsc{SonAngle}(NT) =  $arctan(6-3) - arctan(1-3)$ = 2.35
     \b. \textsc{SonAngle}(TT) =  $arctan(6-1) - arctan(1-1)$ = 1.37
     \c. \textsc{SonAngle}(TN) =  $arctan(6-1) - arctan(3-1)$ = 0.27

The larger the {\sc Sonority Angle}, the larger the faithfulness cost. We therefore expect it to be hardest to epenthesise into NT out of these three clusters, and easiest to epenthesise into TN.

We formalise the idea of the faithfulness cost by defining a family of \textsc{Ident} constraints that penalise outputs that incur faithfulness costs of greater than a certain $n$.

\ex. \textsc{Ident(SonAngle)}$<n$: Assign a violation mark if the consonants in two strings C\textsubscript{1}C\textsubscript{2} and C\textsubscript{1}VC\textsubscript{2} 
stand in correspondence, and the sonority angle between C\textsubscript{1}C\textsubscript{2} and C\textsubscript{1}V is greater than $n$. % TODO: change to it being a property of C1?

These faithfulness constraints have a universal ranking, with the least stringent the highest-ranked.

   \ex. ... \newline
            $\gg$ \textsc{Ident(SonAngle)}$<$1.5 \newline
            $\gg$ \textsc{Ident(SonAngle)}$<$1.0 \newline
            $\gg$ \textsc{Ident(SonAngle)}$<$0.5 \newline
            $\gg$ ...

Note that I do not claim that this is the only dimension of perceptual similarity relevant to epenthesis.  
As we will see later in the paper, other faithfulness constraints may prevent epenthesis, such as a restriction on epenthesising a vowel between two obstruents, which introduces an additional region of sonorance that had previously not existed \cite{flemming.2008}.


\section{Case study: Irish} \label{irish}

\section{Case study: Chaha} \label{chaha}

\section{Issues} \label{issues}

\section{Conclusion} \label{conclusion}

\bibliographystyle{unified}
\bibliography{sonangle}


\end{document}